\section{\textbf{Introduction}} \label{sec:Intro}
	
	Automation of day to day tasks is an endeavour that has moved a large amount of scientific resources in the
	recent history. One specific example is target of studies around the globe by a lot of universities, companies
	and industries, which is the automation of vehicles, more specifically, automobiles. The objective of such
	attempts is the development of artificial intelligences capable of driving a car safely, with traffic law
	enforcement, real-time decision making, efficiency and, in addition, resource economy - as with gas or even
	time. The practical applications of such controllers in autonomous vehicles are truly numerous, for example, the
	researches pursued by DARPA~\cite{DARPA}.

	The Simulated Car Racing Championship (SCRC), using the platform TORCS (The Open Racing Car Simulator), has
	brought an excellent environment for benchmarking AI approaches for the problem of autonomous car controllers.
	Notwithstanding, the optimum behaviour of a controller is a complex matter in its full extent; for this purpose,
	the strategy adopted to deal with it was to divide the problem into smaller portions, i.e., less complicated
	subproblems, in order to implement a finite-state machine that admittedly covers all necessary behaviours. Later
	on this paper, each of those subproblems are treated as the states of the referenced finite-state machine, which
	individually incorporate different but complementary parts of the integral behaviour.
	
	One way to enhance the performance of the created controllers is by comparing each and every possible set of
	parameters or configurations it could assume, but that choice is not always possible due to time and space
	complexities. For that reason, although hand-coded methods might present satisfactory outcomes in questions of
	such intricacy similar to the one discussed in this paper, they will hardly ever outperform the ones that are
	computer-aided on account of the incisive and indefatigable pace that the latter perform the task, which results
	in more possibilities tested. Considering this, after an initial structure of the controller was designed, a
	method of computer-aided fine tuning was assimilated to it, which was a genetic algorithm.
	
	Apart from this Introduction, this paper is structured as follows: Section~\ref{sec:Environment} introduces
	TORCS, the working environment used in the context of this task, along with the competition, SCR Championship,
	that currently represents the utmost metric to evaluate the performance of the controller proposed\toDo{ nao chegamos a correr e utilizar os dados como validacao, apenas usamos os participantes anteriores }, and what is
	already being done at this context in related works, highlighting the strategies that are standing out - State of
	the Art - and analyzing the characteristics responsible for it; Section~\ref{sec:FSM} then explains the proposal
	of the two developed controllers, clarifying their behaviour and structure, and briefly describing how the were
	augmented by the computer-aided method of a genetic algorithm; Section~\ref{sec:Experiments} describes how the
	validation process occurred, through the methodology, the experiments and the results achieved, including their
	correspondent analysis; Section~\ref{sec:Conclusions} provides conclusions about the acquired results, which
	establish the comparison between the finite state machines with few and with moderate number of states, pointing
	out prospects about what is yet to be done in future works to improve those results.
