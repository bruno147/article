\section{\textbf{Introduction}} \label{sec:Intro}
	
	Automation of day to day tasks is an endeavour that has moved a large amount of scientific resources in the recent history~\cite{INDUS}~\cite{APPLI}. One specific example is target of research around the globe by a lot of universities, companies	and industries, which is the automation of vehicles, more specifically, automobiles. The objective of such attempts is the development of artificial intelligences capable of driving a car safely~\cite{SAFE}, with traffic law enforcement, real-time decision making, efficiency and, in addition, resource economy - as with gas, pollution emission~\cite{AUTOM} or even time. The practical applications of such controllers in autonomous vehicles are numerous, for example, the researches pursued by DARPA~\footnote{http://www.darpa.mil/our-research}.
	
	Practical systems are expensive to build effectively, which is the reason why simulations are used for solving this problem. Games present a well defined environment wich may simulate complex situations. More over they provide a large set of possibilities for comparing AI solutions for specific problems such as path planning~\cite{deFreitas:2012}, humam pose recognition~\cite{Shotton:2011}, and others

	The Simulated Car Racing Championship (SCRC), using the platform TORCS (The Open Racing Car Simulator), has	brought an excellent environment for benchmarking AI approaches for the problem of autonomous car controllers~\cite{2009}. Even with the advent of computer simulations, finding the optimum behaviour of a car controller is a complex matter, so, many approaches have been suggested, such as heuristic algorithms~\cite{MrRacer} and modular and fuzzy architectures~\cite{AUTOPIA}. The strategy adopted in this work was to divide the problem into smaller portions, i.e., less complicated subproblems, in order to implement a finite-state machine that admittedly covers all necessary behaviours.

    A finite state machine is a well-known algorithm used as solution in the literature in the attempt to solve several problems. For instance, a state machine was developed to control air volume in air conditioning systems~\cite{BERNARD}. Another example is the surveillance systems in highways for detecting vehicles using a finite state machine~\cite{DOHYUN}. To contextualize of this paper, previous controllers were designed using finite state machines, one of them was designed by Diego Perez, a controller which uses a state machine with fuzzy logic to solve the simulated car racing problem~\cite{DIEGO}.
	
	One way to enhance the performance of the controllers developed is by evaluating each and every possible set of parameters or configurations it could assume, but that choice is not always affordable due to large search spaces. Considering this, after an initial structure of the controller was designed, a method of computer-aided fine tuning was assimilated to it, which was a genetic algorithm.
	
	The rest of this paper is structured as follows: Section~\ref{sec:Environment} introduces TORCS, the working environment used in the problematic presented, along with the competition, SCR Championship, that currently represents the metric to evaluate the performance of controllers proposed for this environment; the section also presents what is already being done at this context in related works. Section~\ref{sec:FSM} then explains the proposal of the two developed controllers, clarifying their behaviour and structure, and briefly describing how they were augmented by the computer-aided method of a genetic algorithm. Section~\ref{sec:Experiments} describes how the validation process occurred through the methodology, the experiments and the results achieved, including their correspondent analysis. Section~\ref{sec:Conclusions} provides conclusions about the results acquired, which establish the comparison between the finite state machines with few and with moderate number of states, pointing out prospects about what might be done in future works to improve those results.
