\section{Conclusions}\label{sec:5}
Self-driving cars are one of the most interesting promises of technology, but the process of developing a software controller is complex and expensive. Advanced car racing simulators present a well defined environment for developing such drivers and comparing Artificial intelligence solutions, and The Open Racing Car Simulator/Simulated Car Racing Championship platforms have been successfully used to this end.

This work presented two versions of finite-state machine controllers, an interesting approach that clearly breaks the task into distinct, independent behaviors (states), facilitating testing different configurations and simplifying the search for better configurations. This search was executed by applying a genetic algorithm with default configurations found in the literature to evolve the controllers' parameters in the proposed models. The evolved controller was submitted to the 2015 Simulated Car Racing Championship and finished in 4th place.

Experimental results showed that profiling the evolution process according to the track type has significant impact on the performance. The 5-state model had an overly complex implementation, in which the transition function hampered the evolution of the states parameters. The 3-state controller had a simple, intuitive model which was performed well in races, surpassing one of AUTOPIA's marks which could result in it qualifying ahead of the current state of the art in SCR Championship's racing stage.

The learning module potentially leads to better drivers when damage is a concern, with small impact on the performance. However, its behavior handling sharper curves needs further investigation. Other ways to improve the controllers endurance are longer periods in the fitness function and integration of harsher brake policies or path planning to avoid the car leaving the track. Additionally, a better breaking system, such as anti-lock breaking system or traction control system, could lead to improved performance by reducing skidding and providing more stable breaking and curve handling.

Considering the offline learning, different techniques could be investigated and compared to the GA's results. The genetic algorithm could also be improved through a fine tuning of its parameters, and a less general approach to representing a solution for the model (replacing floating point representations by parameter specific ones), or even incorporating fuzzy logic to handle ordering issues with some parameters could simply solve this ordering issue.

Finally, further development on handling opponents must be pursued as focus must be set on developing skills for the racing stage of the championship, specially for dealing with opponents.