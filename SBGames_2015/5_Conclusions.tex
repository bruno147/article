\section{Conclusions}\label{sec:5}
Self-driving cars are one of the most interesting promises of technology, but the process of developing a software controller  is complex and expensive. Advanced car racing simulators present a well defined environment for developing such drivers and comparing Artificial intelligence solutions for specific problems such as path planning~\cite{deFreitas:2012}, controlling human non-playable character~\cite{simon2008}, car racing~\cite{2009}, among others.

The Open Racing Car Simulator (TORCS) is a modern, modular, highly-por

also complex since testing is a frequent task and real driving systems are expensive. Creating and testing solutions can be aided by realistic car racing games which can closely simulate the roads, other vehicles, and their complex interactions~\cite{caldeira2013torcs}. Games also present a well defined environment which may be used not only for applications of machine learning results, such as neuroevolution~\cite{stanley_real-time_2005,5482132} or human pose recognition~\cite{Shotton:2011}, but also for comparing AI solutions for specific problems such as path planning~\cite{deFreitas:2012}, controlling human non-playable character~\cite{simon2008}, car racing~\cite{2009}, among others.

The Open Racing Car Simulator (TORCS) is a modern, modular, highly-portable multi-player, multi-agent car simulator~\cite{SIMUTORCS}, one of the most advanced racing games available, and frequently used as a platform for comparing driver solutions in the Simulated Car Racing (SCR) Championship~\cite{2009,Loiacono:2012:LEA:2212908.2212953}. This paper uses a finite-state machine (FSM) controller to exploit the advantages of a divide-and-conquer approach by considering the task's various situations as distinct states (racing, getting back on track, etc.).

FSM is a well-known mathematical model with widespread applications such as controlling air conditioning systems~\cite{BERNARD}, highway surveillance systems~\cite{DOHYUN}, and even simulated car racing~\cite{DIEGO}. Such solutions are implemented as software and configured according to a specified set of parameters, and defining the best possible configuration is usually a huge combinatorial problem for which exhaustive systematic searches become unfeasible.

Autonomous vehicles have many practical applications, however the development of software controllers for such task is difficult. This work presents a finite state-machine model with evolved parameters as a suitable solution for a self-driving car, and the comparison of two different configurations in The Open Racing Car Simulator. This approach enables a clear division of behaviors in states, providing an easy way to test different configurations and simplifying the search for better controllers by allowing changes in selected states. A 5-state and a 3-state drivers were evolved through genetic algorithm and compared to each other and to AUTOPIA, the current state of the art controller for the Simulated Car Racing Championship. Results showed that the proposed model has potential for racing, even surpassing one of AUTOPIA's marks, and provide insights on developing and configuring.

	This paper proposed two approaches developed to control a car during a race in a simulated computational environment, the game platform TORCS. The models of both of these controllers were described, explained, enhanced by means of a genetic algorithm, compared and then tested together with a State of the Art controller - AUTOPIA. It was implied before the testing phase that a finite state machine too burdened in the process of transition between states might lose performance, which was corroborated by the experimental results of the comparison of the two models detailed.

\subsection{Conclusions} \label{subsec:Conclusions}

	A veil was put over the Five-State FSM from the very beginning of the experimentation phase, which was its great dependency towards the transition function. Early superficial evaluations of the performance of this first model indicated an overcharge concerning this function, which was verified by considerable changes in behaviour derived from adjustments in its parameters. For that reason, a second model with less states - the Three-State FSM - was designed regarding this characteristic and releasing part of the performance burden from the transition function.

	The finite state machine with less states achieved a superior overall performance in the tests carried out, in relation to the one with more states. The simplifications fashioned in the transition function of the former were inferred to be the reason for this improvement, along with its intricate relation with the number of parameters that were target of fine tuning in the evaluation and validation process.

	The evolution procedure adopted concerning the controllers culminated in three characteristic behaviors. The controller evolved on road tracks became a fast driver, whose hastiness resulted in a careless attitude in general; in other words, it was only good for races in road tracks. The one evolved on dirt tracks turned out to be too careful in contrast, limitedly determining its speeds and, in efficiency terms, inferior. The controller evolved on a mixed set of tracks inherited characteristics from both the previous ones, becoming swift but not too hasty, prudent but not too slow. The latter surpassed the performance of the dirt-evolved drivers even on dirt tracks, and did not lose by far on road tracks in comparison to drivers evolved solely in them.

	In terms of speed, the Three-State FSM was able to overcome AUTOPIA in one of the three tracks used to validate the drivers using the distance covered in 10 000 game tics as metric. However, in comparison to the same controller, using the time elapsed to race 10 laps as metric instead, the experimental results provided an insight on the lack of endurance that the finite state machine drivers proposed possess.

	To sum up, the interpretation of the global results from the experiments performed gives margin to declare that finite state machines are a reliable technique to implement artificial intelligences, at least for computer games such as simulated races. They provide the possibility of parallel development and also enable parameter tuning in separate fronts, due to the independence and abstraction between the behaviors from each state. Finite state machines also represent a valuable tool for describing an operation model for a process, simplifying and gathering possible situations it might present into straightforward categories of accessible understanding.

\subsection{Future Works} \label{subsec:Future}

	One characteristic that has been marked as a deficiency in the controllers presented in this paper is the lack of endurance. In order to prevent this from affecting the general performance of the controllers developed, more robust techniques must be integrated into their model. Ways of treating this matter range from harsher brake policies to drive planning intensification, which are already being taken into account for future proposals.

	Anti-lock Breaking System (ABS) filter is commonly used in the racing environment~\cite{5593318} and was not implemented in the proposed controller. The Traction Control System (TCS)~\cite{5593318} could also be implemented by applying filters to the acceleration actuator This policies minimize skidding and increase the stability when speed needs to be adjusted in a turn.

	The driver uses some parameters that are necessarily greater than others. A model based on fuzzy logic \cite{DIEGO} could simply solve this ordering issue.

	Another important task to be accomplished is the opponent treatment in real-time races. Routines to reduce collisions, i.e., avoiding being overtaken and also being concerned about overtaking the opponents is a fundamental issue. Ignoring adjacent cars usually causes the driver to face unexpected collisions, ending up stuck, considering a worst case scenario. Many of the renowned developers for TORCS already incorporate such treatment in their controllers, and neglecting this necessity renders any driver less robust to unexpected race events, and also reduces its performance.

