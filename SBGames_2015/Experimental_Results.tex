\section{Experimental Results} \label{sec:Experiments}

	A veil was put over the Five-state FSM from the very beginning of the experimentation phase, which was its great
	dependency towards the transition function. Early superficial assessments of the performance of this first model
	indicated an overcharge concerning this function, which was verified by considerable changes in behaviour derived
	from adjustments in its parameters. For that reason, a second model with less states - the Three-state FSM - was
	designed regarding this characteristic and releasing part of the performance burden from the transition function,
	and the results from the comparison of these architectures are described and analyzed in the succeeding
	subsections.

\subsection{Methodology} \label{subsec:Methodology}

	Once defined the models and which of their parameters required tuning, the genetic algorithm was applied to each
	one separately, in order to adjust their configurations for superior and competitive status. At the same time, the
	goal was to find general and versatile controllers with good results for any track and specific ones that were
	fittest to race in various tracks: road, dirt and oval alike. Therefore, due to the differences observed in the
	parameters of controllers evolved on dirt and road tracks, the evolution process took place separately for those
	two kinds of environments, which produced two contrasting sets of enhanced values for each model.
	
	The \emph{metric} chosen to evaluate the generated controllers was the combined sum of the distance raced by the
	car alone in the first 10.000 game cycles - also called \emph{game tics} - of a list of mixed tracks. This value
	will henceforth be called the \emph{fitness} of the controller, as it was used to determine whether he would
	remain in the evolution process.
	
	The experiments concerning Oval Tracks would repeatedly provide inconclusive results, so they were neglected in
	this evaluation process. Thus, in order to find the best set of parameters for a general track, the two
	finite-state states were evolved in three different sets of tracks, one with 4 Dirt Tracks, one with 4 Road
	Tracks and another with the 4 of each type. The evolution process for each set of tracks consisted on 600
	generations of 30 individuals and culminated in one controller; in other words, At the end of the experiments,
	there were six evolved controllers, one specific for Road Tracks for each model, one specific for Dirt Tracks for
	each model, and also one evolved in a mixed manner for each model.
	
	The validation occurred through testing the six produced pilots in a predefined set of tracks different from those
	in which they were evolved, lest they would evidence too track-limited parameters. On behalf of comparison, the
	results from the AUTOPIA controller were incorporated in the analysis, since it can be considered the State of
	the Art, displaying one of the best performances for the SCR Championship, and should provide satisfactory basis
	for appraisal.
	% the set of tracks was defined by autopia
	% tracks image as in the autopia's article
	% explain warm up stage
	
\subsection{Results} \label{subsec:Results}

	% is the fitness graphic among generations appropriate here?
	% add tables with the evaluation results
	
\subsection{Analysis} \label{subsec:Analysis}
	
	
	