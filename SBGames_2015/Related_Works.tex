\section{\textbf{Related Works and State of the Art}} \label{sec:Related}
	
	It is very common among some of the SCRC awarded controllers the incorporation of machine learning in their driving methods, along with other evolving techniques using artificial intelligence~\cite{2009}. Due to the uncertainty of the environment, track is initialy unknown to the controller, learning procedures may grant some advantage, enhance performance and competitiveness. Essentially, there are two ways of evolving learning solutions: Online Learning and Offline Learning.

	According to Tom Mitchel in \emph{Machine Learning}~\cite{Mitcchel:ML}:

	\begin{quotation} \itshape

		``Systems that learn by moving about the real environment and observing the results are typically called online systems, whereas those that earn solely by simulating actions within an internal model are called ofline systems.''
	
	\end{quotation}

		The current champion of the SCR Championship is the controller \emph{Mr. Racer}~\cite{MrRacer}, and it has proven to be the State of the Art by winning the last three competitions that happened from 2011 to 2013. The authors of this implementation employ several heuristics and black-box optimization methods in order to reproduce the mechanisms to which human racing drivers resort, doing so by means of a modular structure, which means it is builded oriented at modules, normally focused at the actuators. For instance the \emph{AUTOPIA} at SCRC 2009 ~\cite{AUTOPIA2009} has six modules: gear control, target speed, speed control, learning and oponent managements. \emph{Mr. Racer} uses a Covariance Matrix Adaptation Evolution Strategy (CMA-ES), to evolve parameters offline.
	
	According to the founders of the competition~\cite{SCRC} and the authors of \emph{Mr. Racer} themselves, \emph{AUTOPIA}~\cite{AUTOPIA} is another competitive controller, with the potential to even be the best one available. \emph{AUTOPIA} implements a modular Fuzzy Architecture, whose division contains gear, steering and speed control; and it is optimized by means of a genetic algorithm for Offline Learning, and by means of landmarking the lane exit points for further speed reduction for Online Learning.

	Some solutions regarding the development of a controller used neural networks, such as \emph{Luigi Cardamone}~\cite{exp}, which used imitation learning, and \emph{Jorge Mu\~{n}oz}~\cite{Munoz}, which work attempt to create a human-like controller.
	
	These and other controller exemplifications~\cite{SCRC} served as criteria for the analysis and development of the approach presented in this paper. Elements incorporated and adapted from them feature modularity, Offline Learning through genetic algorithms, Online Learning through landmarking and choosing sets of parameters for different categories of tracks, etc.
	
