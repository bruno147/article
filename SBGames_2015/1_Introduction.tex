\section{Introduction}\label{sec:1}

Automation of day to day tasks is an endeavor that has moved a large amount of scientific resources in the recent history~\cite{INDUS,APPLI}. One of the more desirable goals is the advent of self-driving cars, which should drive safely and efficiently, within traffic laws~\cite{SAFE,AUTOM}, bringing hope to current issues of traffic in urban roads by aiming at shorter response times, better fuel consumption, and lower levels of pollution. Such drivers, however, face several difficult challenges such as perception, navigation and control~\cite{6179503}.

Another critical issue is driver development, since testing is a frequent task and real driving systems are expensive. Creating and testing solutions can be aided by realistic car racing games which can closely simulate the roads, other vehicles, and their complex interactions~\cite{caldeira2013torcs}. Games also present a well defined environment which may be used not only for applications of machine learning results, such as neuroevolution~\cite{stanley_real-time_2005,5482132} or human pose recognition~\cite{Shotton:2011}, but also for comparing AI solutions for specific problems such as path planning~\cite{deFreitas:2012}, controlling human non-playable character~\cite{simon2008}, car racing~\cite{2009}, among others.

The Open Racing Car Simulator (TORCS) is a modern, modular, highly-portable multi-player, multi-agent car simulator~\cite{SIMUTORCS}, one of the most advanced racing games available, and frequently used as a platform for comparing driver solutions in the Simulated Car Racing Championship (SCR)~\cite{2009,Loiacono:2012:LEA:2212908.2212953}. This paper uses a finite-state machine (FSM) controller to exploit the advantages of a divide-and-conquer approach by considering the task's various situations as distinct states (racing, getting unstuck, getting back on track, etc.).

FSM is a well-known mathematical model with widespread applications such as controlling air conditioning systems~\cite{BERNARD}, highway surveillance systems~\cite{DOHYUN}, and even simulated car racing~\cite{DIEGO}. Such solutions are implemented as software and configured according to a specified set of parameters, and defining the best possible configuration is usually a huge combinatorial problem for which exhaustive systematic searches become unfeasible. Thus, several machine learning approaches have been applied to it (heuristic algorithms~\cite{MrRacer}, evolutionary algorithms~\cite{Nallaperuma:2014}, modular fuzzy architectures~\cite{AUTOPIA}, sensory-to-motor couplings~\cite{COBOSTAR}, among others). One of the most successful optimization methods is Genetic Algorithm (GA), which is has been applied to a myriad of problems such as non-linear systems identification~\cite{GACTRL}, biomedicine prosthesis development~\cite{GABIO}, agent behavior forecasting~\cite{GAECO}, improving classifiers~\cite{pedrycz_genetic_2005}, and others.

This work uses a GA for searching for optimal parameter settings for the proposed FSM based driver, using TORCS as test bed. The rest of this paper is structured as follows: Section~\ref{sec:2} introduces the TORCS/SCR, finite-state machines, and Genetic Algorithms concepts involved; Section~\ref{sec:3} details the proposed controller model. Section~\ref{sec:4} describes the evaluation and validation process; and Section~\ref{sec:5} presents concluding remarks.
