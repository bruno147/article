\begin{table}
	\renewcommand{\arraystretch}{1.3}
	\caption{Parameters evolved}
	\centering
\begin{tabular}{|c||c|}
\hline
\bfseries Parameter & \bfseries Description \\
\hline
\hline STUCK\_SPEED &  Threshold velocity for transitioning into the stuck state. The driver will be considered stuck when running bellow that velocity for a certain number of ticks. \\
\hline MAX\_SLOW\_TICKS & Maximum number of ticks running in low speed before the driver is considered stuck.\\
\hline MAX\_STUCK\_TICKS & Maximum number of ticks that the driver will perform the reverse behavior. \\
\hline MIN\_RACED\_DIST &  Minimal distance required for the stuck verification to begin. This avoids wrong stuck detections at the very beginning of the race.\\
\hline LOW\_GEAR\_LIMIT &  Threshold for considering current gear as low gear.\\
\hline LOW\_RPM & Threshold for considering current RPM as low. \\
\hline AVERAGE\_RPM & Threshold for considering current RPM as average. \\
\hline HIGH\_RPM & Threshold for considering current RPM as high. \\
\hline SPEED\_FACTOR & Scales the difference among sensor in order to obtain the target speed. \\
\hline BASE\_SPEED & Offset velocity value. \\
\hline MAX\_SKIDDING & Maximum skidding allowed before any recover action is taken in the out of track state. \\
\hline VELOCITY\_GEAR\_4 & Threshold velocity for changing to fourth gear in the OutOfTrack state. \\
\hline VELOCITY\_GEAR\_3 & Threshold velocity for changing to third gear in the OutOfTrack state. \\
\hline VELOCITY\_GEAR\_2 & Threshold velocity for changing to second gear in the OutOfTrack state. \\
\hline NEGATIVE\_ACCEL\_PERCENT & Reduces the acceleration based on how much the car is skidding. \\
\hline MAX\_RETURN\_ANGLE & Maximum angle desired for reentering the track. \\
\hline MIN\_RETURN\_ANGLE & Minimum angle desired for reentering the track. \\ 
\hline MAX\_STRAIGHT\_LINE\_VAR & Upper variance threshold for the driver to be considered in a straight line. \\ 
\hline MIN\_STRAIGHT\_LINE\_VAR & Lower variance threshold for the driver to be considered in a straight line. \\ 
\hline MAX\_APPROACHING\_CURVE\_VAR & Upper variance threshold for the driver to be considered approaching a curve. \\
\hline MIN\_APPROACHING\_CURVE\_VAR & Lower variance threshold for the driver to be considered approaching a curve. \\
\hline 

\end{tabular}

\end{table}

The 4 last parameters are only present in FSM5