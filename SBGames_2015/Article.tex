\documentclass[a4paper]{sbgames}               % final
%\usepackage[scaled=.92]{helvet}
\usepackage{times}
\usepackage{graphicx}

%% use this for zero \parindent and non-zero \parskip, intelligently.
\usepackage{parskip}

%% the 'caption' package provides a nicer-looking replacement
\usepackage[labelfont=bf,textfont=it]{caption}

\usepackage{url}

\usepackage{xcolor}
\newcommand{\toDo}[1]{\textcolor{red}{#1}}
\newcommand{\gnramos}[1]{\textcolor{blue}{#1}}

%% Paper title.
\title{Comparing the Performance of Finite-State Machines with Different Numbers of States on TORCS}

%% Author and Affiliation (multiple authors). Use: and between authors

\author{Bruno H. F. Macedo\\Gabriel F. P. Araujo\\
        \and Gabriel S. Silva\\Matheus C. Crestani\\\textit{University of Bras\'{i}lia}
        \and Yuri B. Galli\\ Guilherme N. Ramos
}

\contactinfo{\{bhfmacedo,yurigalli\}@gmail.com \\
             gnramos@unb.br \\ \gnramos{so um contato - do autor correspondente}
}
%% Keywords that describe your work.
\keywords{finite-state machine, computer games, TORCS, SCRC, artificial intelligence, genetic algorithm, self-driving car}

%% Start of the paper
% Attention: As you need to insert EPS images in Postscript,
% you need to insert PDF images into PDFs.
% In the text, extensions cancbe omitted (latex use .eps, pdflatex get .pdf)
% To convert them: epstopdf myimage.eps
\begin{document}

%\teaser{
%  \includegraphics[width=\linewidth]{sample.pdf}
%  \caption{Optional image}
%}

%% The ``\maketitle'' command must be the first command after the
%% ``\begin{document}'' command. It prepares and prints the title block.

\maketitle

%% Abstract section.

	\begin{abstract}

		\toDo{Alterar cabecalho do programa para aceitar portugues}
		\gnramos{para que? (o texto � em Ingles - e poderia ser em ingl�s mais simples/objetivo) - to-do: \textbf{escrever 10 paginas}}

		\gnramos{reescrever. abstract deveria ter $\pm$ 5 frases objetivas: 1) problema 2) proposta 3) m�rito 4) valida��o 5) resultados}
		This work presents two different approaches developed to control a self-driving car in the racing environment
		simulated by TORCS. The problematic revolves around the representation of the complex behaviour that
		describes a pilot during a race, whose \gnramos{refere-se ao comportamento?} proposed solutions are two different finite-state machines; while one
		method treats the problem with a more partitioned approach, the other tries to generalize the possible
		situations. Furthermore, in order for the final controller to attain competitive status, i.e. parameter fine
		tuning \gnramos{explicacao desnecessaria}, the	incorporation of a genetic algorithm was performed\gnramos{``foi feita a incorpora��o de GA''? melhor ser objetivo: ``utilizou-se GA para'' }. Evolving the driver using this method
		initially required the definition of a set of parameters, where the optimization lied effectively\gnramos{o que isso quer dizer? (lied effectively)}. Validation
		occurred through exhaustively \gnramos{n�o foi exaustivo} testing the configurations of the two approaches, comparing them along with
		artificial intelligences provided by the platform and also State of the Art implementations\gnramos{tantos detalhes n�o s�o necess�rios}.

	\end{abstract}

	%% The ``\keywordlist'' command prints out the keywords.
	\keywordlist
	\contactlist

%	\section{Topics}%
%
%	Talk about:
%
%	\begin{itemize}
%		\item How the problem appeared;
%		\item The approach chosen to solve it;
%		\item How we evolved it by applying a genetic algorithm;
%		\item How we improved the process of evolving through virus infection.
%	\end{itemize}
%
%	Include:
%		\item State-of-art exposure;
%		\item Model definition (theoretical outline);
%		\item Test results;
%		\item Results analysis;
%		\item Prospects;
%		\item Future works.
%	\end{itemize}

	\section{\textbf{Introduction}} \label{sec:Intro}
	
	Automation of day to day tasks is an endeavour that has moved a large amount of scientific resources in the recent history~\cite{INDUS}~\cite{APPLI}. One specific example is target of research around the globe by a lot of universities, companies	and industries, which is the automation of vehicles, more specifically, automobiles. The objective of such attempts is the development of artificial intelligences capable of driving a car safely~\cite{SAFE}, with traffic law enforcement, real-time decision making, efficiency and, in addition, resource economy - as with gas, pollution emission~\cite{AUTOM} or even time. The practical applications of such controllers in autonomous vehicles are numerous, for example, the researches pursued by DARPA~\footnote{http://www.darpa.mil/our-research}.
	
	Practical systems are expensive to build effectively, which is the reason why simulations are used for solving this problem. Games present a well defined environment wich may simulate complex situations. More over they provide a large set of possibilities for comparing AI solutions for specific problems such as path planning~\cite{deFreitas:2012}, humam pose recognition~\cite{Shotton:2011}, and others

	The Simulated Car Racing Championship (SCRC), using the platform TORCS (The Open Racing Car Simulator), has	brought an excellent environment for benchmarking AI approaches for the problem of autonomous car controllers~\cite{2009}. Even with the advent of computer simulations, finding the optimum behaviour of a car controller is a complex matter, so, many approaches have been suggested, such as heuristic algorithms~\cite{MrRacer} and modular and fuzzy architectures~\cite{AUTOPIA}. The strategy adopted in this work was to divide the problem into smaller portions, i.e., less complicated subproblems, in order to implement a finite-state machine that admittedly covers all necessary behaviours.

    A finite state machine is a well-known algorithm used as solution in the literature in the attempt to solve several problems. For instance, a state machine was developed to control air volume in air conditioning systems~\cite{BERNARD}. Another example is the surveillance systems in highways for detecting vehicles using a finite state machine~\cite{DOHYUN}. To contextualize of this paper, previous controllers were designed using finite state machines, one of them was designed by Diego Perez, a controller which uses a state machine with fuzzy logic to solve the simulated car racing problem~\cite{DIEGO}.
	
	One way to enhance the performance of the controllers developed is by evaluating each and every possible set of parameters or configurations it could assume, but that choice is not always affordable due to large search spaces. Considering this, after an initial structure of the controller was designed, a method of computer-aided fine tuning was assimilated to it, which was a genetic algorithm.
	
	The rest of this paper is structured as follows: Section~\ref{sec:Environment} introduces TORCS, the working environment used in the problematic presented, along with the competition, SCR Championship, that currently represents the metric to evaluate the performance of controllers proposed for this environment; the section also presents what is already being done at this context in related works. Section~\ref{sec:FSM} then explains the proposal of the two developed controllers, clarifying their behaviour and structure, and briefly describing how they were augmented by the computer-aided method of a genetic algorithm. Section~\ref{sec:Experiments} describes how the validation process occurred through the methodology, the experiments and the results achieved, including their correspondent analysis. Section~\ref{sec:Conclusions} provides conclusions about the results acquired, which establish the comparison between the finite state machines with few and with moderate number of states, pointing out prospects about what might be done in future works to improve those results.

	\section{\textbf{The Simulation Environment}} \label{sec:Environment}

	Multiple applications that require verification and evaluation have become dependent of expensive resources, which makes their development a delicate matter. Examples of systems whose tests involve costly supplies are medical education~\cite{MEDIC}, the aviation industry~\cite{AIR} and automotive research and development~\cite{AUTR}; because naturally no company or enterprise has expendable numbers of bodies for tests, or airplanes and cars to crash in order to safe-proof and improve. In these conditions computer simulations arise, virtually reproducing the environments of the said and of other important applications so that their testbeds reduce the need of the scarce resources related to them.
	
	Many other reasons justify the use of virtual simulations, varying from one application to another, and, because of this, countless companies are incorporating them in their educational and training process~\cite{Simulation}~\cite{Operator}. A few number of reasons why a simulation was used in this work:
	
	\begin{itemize}
		
		\item managing control operations;
		
		\item planning actions;
		
		\item understanding a problem and finding out how to react to it;
		
		\item training and learning through experimentation.
		
	\end{itemize}
	
	Simulations represent just an approximated model of the systems they try to imitate, but in most cases the effects of the estimates adopted in the attempt to do this do not come to the point that the model becomes invalid. Similarly, car racing simulations also incorporate flawed modeling - regarding friction, gravity, trajectory lines, and so on - but resorting to them may save some spare parts that would most likely be jeopardized in real-life tests. So, new studies, new researches and new automation methods might be evaluated through simulations and avoid costs in the development process.
	
\subsection{TORCS} \label{subsec:TORCS}

	The Open Racing Car Simulator (TORCS) is a platform that is renowned for its highly credible physics modeling engine and yet user-friendly interface for car racing simulations~\cite{TORCS}. It is a modern, multi-player and multi-agent car simulator; it also is an interface that is widely used for benchmarking AI~\cite{2009} due its high degree of modularity and portability, concerning multi-platform environments and support to the programming languages C and C++. Artificially intelligent agents can be developed as modules inside TORCS where there are many possible levels of abstraction. For example, at the car level, there is an intelligent control system for each car component; at the driver level, mid-level control systems for complex driving agents could be implemented using the partial simulation information given by a low-level API provided by TORCS.

	The TORCS engine~\cite{SIMUTORCS} uses a discrete-time simulation, whose discretization is set to approximately 0.002s of simulation time, and engine solves differential equations using \"{E}uler method, and all basic elements of the vehicular dynamics are handled, which are:

	\begin{itemize}
		
		\item basic properties of the vehicular system;
		
		\item mechanical details;
		
		\item dynamic and static friction;
		
		\item aerodynamic model.
		
	\end{itemize}

	Mass, rotational inertia of the car, engine, wheels, and other components, are included in the model of the vehicular system; while the types of different suspension, links, and differentials are done so in the mechanical model. The profiles for different ground types with both dynamic and static friction are also included; this way, the aerodynamics modeling includes slipstreaming and ground effects, that vary from one profile to another. Nevertheless, the simulation engine can be replaced or easily modified as a result of the modularity supplied by TORCS. The interface with this platform occurs by means of a sensor-based interaction system in which the developer is able to interpret received parameters of the car - such as speed in X, Y and even Z axes - and control the car through programming its actuators, some of which are acceleration and steering.
	
	Another credibility factor for this platform is its non-punctual cars that interact with each other in the races by a life-like collision system. Still, TORCS is still a simulator, and its limitations, along with the defined racing environment and the modeled car, are more than likely to affect any results obtained. This is an inherited characteristic of any real-life problem simulation, what in academia is denominated \emph{reality gap}~\cite{RG}, and it stems from the simplifications made concerning the car models, the technical features of the tracks, and so forth.
	
	The SCR interface occurs in a way that TORCS allows the controller to have a full view of the environment including its exact location inside the track, the geometry and friction and also the exact location of all the other cars. In order to separate the controller logic from the simulator, the SCR interface was designed~\cite{2009}. This abstraction limits the information received by the controller, and the communication between it and TORCS is made through a client-server interface, with each player receiving information from the server regarding the sensors of the car, and in return providing actuator values that determine how the controller is supposed to drive the car.

\subsection{The SCR Championship} \label{subsec:SCRC}

	The Simulated Car Racing Championship (SCRC) is an example of a well-known competition which utilizes TORCS as simulator~\cite{SCR}. Being an event joining three competitions held at major scientific conferences, such as \emph{IEEE Congress on Evolutionary Computation}~\footnote{http://www.cec2015.org/}, \emph{Genetic and Evolutionary Computation Conference}~\footnote{http://www.sigevo.org/gecco-2015/} and \emph{IEEE Conference on Computational Intelligence and Games}~\footnote{http://www.ieee-cig.org/}, it is an accepted metric of evaluation in the fields of Evolutionary Computation and Computational Intelligence regarding games in general.

   	\begin{figure}[h]
   		
		\centering
		\includegraphics[width=250pt]{Figure1.jpg}
		\caption{Available data inside TORCS becoming data accessible to the client}
		\label{Fig:Comm}
		
	\end{figure}

	Race tracks are categorized into \emph{Road}, \emph{Dirt} and \emph{Oval} inside TORCS. The races from the SCRC take place in track types decided by the organization of the championship, information which is not provided to the participants and that may incorporate maps that are unknown to them. The competition adopts a structure that gathers a \textit{Warm-up} stage, a \textit{Qualifier} stage and a \textit{Final} race. Noise can be introduced in the sensors, option that is present during the actual competition. The complete sensorial input information and all the details concerning the race stages and types are presented at the Simulated Car Racing Championship Competition Software Manual~\cite{SCRC}.
	
	The reason why TORCS presents itself as a satisfactory AI benchmark, in combination with SCRC, is because even	though there are multiple possibilities on how the sensorial input received from the server can be translated into the behavior of the actuators, they can all be compared in a race, which has a robust and steady scoring and evaluational system. In other words, there are many different approaches concerning how to teach the racer encoded by the developers to drive in a racing competition only with the information given by the sensors, and the metric to that issue is the performance on the race itself.


	%\section{\textbf{Related Works}} \label{sec:relworks}
	
		The reason why TORCS presents itself as a satisfactory AI benchmark is because there is an infinity of
		possibilities on how the sensorial input received from the server will be translated into the behaviour of the
		actuators, and they can all be compared in a race, which has a robust and steady scoring system. In other
		words, there are many different approaches concerning how to teach the racer encoded by the developers to
		drive in a racing competition only with the information given by the sensors, and the metric to that issue
		is the performance on the race itself.
		
		By controller, let it be understood that the subject is the programming code that in fact controls the
		car/driver/racer within racing environment. Some examples of awarded controllers and their driving methods
		will be presented in this section. They are what can be called the State of the Art among TORCS, and it is
		very common among them the incorporation of machine learning methods, along with other evolving techniques
		using artificial intelligence. Instinctively, as the nature of the problem comprises evolution by experience,
		learning procedures tend to enhance performance and competitiveness. Essentially, there are two ways of
		evolving controllers: Online Learning and Offline Learning, the first meaning that improvements are achieved
		during the actual race execution time and the latter that it is done before the competition, on the account of
		the developers themselves and with their own resources.
		
\subsection{State of the Art}
		
		The current champion of the SCR Championship is the controller \emph{Mr. Racer}~\cite{MrRacer}, and it has
		proven to be the State of the Art by winning at least the last three competitions that happened. The authors
		of this implementation learn parameters offline through Covariance Matrix Adaptation Evolution Strategy
		(CMA-ES), use regression and low-pass filtering to reduce noise impact, distinguish normal asphalted roads
		from dirt-based ones for behavioral separation and implement an authentic opponent-handling method. Their
		Online Learning consists on the track model selection to categorize into dirt or asphalt, choice of databased
		sets of parameters that best fit the track and the tuning of a target speed for all its corners.
		
		Another renowned controller is \emph{AUTOPIA}~\cite{AUTOPIA}. According to the founders of the
		competition~\cite{SoA} and the authors of \emph{Mr. Racer} themselves, it is a competitive match, with the
		potential to even be the best one available, but since no entries were received from them in a while, their
		means of winning a competition were somewhat restrained. Nevertheless, assessing its performance is
		worthwhile, and its description is the implementation of a modular Fuzzy Architecture, whose division contains
		gear, steering and speed control. Their controller is optimized by means of a genetic algorithm for Online
		Learning, and by means of landmarking the lane exit points for further speed reduction for Offline Learning.\toDo{verificar os termos offline e online}
		
		\toDo{Nós vamos	comparar o nosso piloto com o AUTOPIA? Se sim, devemos dizer isso aqui.}
		
		These and other controller exemplifications served as parameters for the analysis and development of the
		approach presented in this paper. Aspects incorporated and adapted feature modularity, offline learning
		through genetic algorithms, online learning through landmarking and choosing sets of parameters for different
		categories of tracks, etc. \toDo{Adicionar o quê mais fizemos no projeto como pincelada inicial para fazer o
		gancho com a seção Controller Structure.}
	\section{\textbf{The FSMDriver}} \label{sec:FSM}
	
	According to Mat Buckland in \emph{Programming Game AI By Example}~\cite{Buckland:2005:AI}:
	
	\begin{quotation} \itshape
		
		``A finite state machine is a device, or a model of a device, which has a finite number of states it can be in at any given time and can operate on input to either make transitions from one state to another or to cause an output or action to take place. A finite state machine can only be in one state at any moment in time.''
		
	\end{quotation}
	
	An architecture following this guideline was chosen in order to transform the problem of complex driving into smaller problems that describe the situations found within the racing environment.
	
\subsection{The Design of the Behavior States}
	
	Initially, the design of the finite state machine proposed comprised the following states:
	
	\begin{itemize}

	\item \emph{Straight Line};
	
	\item \emph{Approaching Curve};
	
	\item \emph{Curve};
	
	\item \emph{Out of Track};
	
	\item \emph{Stuck}.

	\end{itemize}
	
	Essentially, for this first method, normal behavior covered \emph{Straight Line}, \emph{Approaching Curve} and \emph{Curve}, as the controller was located inside the track boundaries, whereas exception behavior consisted of \emph{Out of Track} and \emph{Stuck}, situations in which recovery actions are expected.
	
	The main difference between the method described and the second one is the way they deal with normal behavior, while one separates it into \emph{Straight Line}, \emph{Approaching Curve} and \emph{Curve}, the other treats it as a unique conduct in the form of the \emph{Inside Track} state. Therefore, the modified finite-state machine was composed by only three states, which were:
	
	\begin{itemize}
		
		\item \emph{Inside Track};
		
		\item \emph{Out of Track};
		
		\item \emph{Stuck}.
		
	\end{itemize}
	
\subsubsection{Normal Behavior}
	
	If a controller using the first method was currently in \emph{Straight Line}, it would be expected of him to simply go as fast as he could, maitaining itself parallel to the track axis. When in \emph{Approaching Curve} state, he would reposition himself and achieve a certain calculated target speed proportionally to the curvature of the approaching curve in order to achieve higher speeds once inside of it. For example, if there is a sharp left turn nearby, the controller sets a target steer that the car needs to obtain before entering it, while bringing the car further to the right of the lane; that way, when the left turn comes, the car can proceed with a less abrupt change in steering, which as a consequence results in a higher speed. Otherwise, when in situations of \emph{Curve}, the pilot would stop braking while maintaining the steering direction towards the sensor pointing the biggest distance value read - which represents the direction of the curve, prepared by the approaching curve state. Figure~\ref{Fig:FSensor} exhibits a car receiving information from the sensors, with the biggest vector being the direction of the curve to be entered.
	
	\begin{figure}[h]
		
		\centering
		\includegraphics[width=250pt]{FarthestSensor}
		\caption{Sensorial input indicating a curve through the biggest value received}
		\label{Fig:FSensor}
		
	\end{figure}
	
	\emph{Inside Track}, therefore, is how the car, desirably, will spend most part of the time, if the second method is being employed. The controller calculates a \emph{target speed} based on how far the car is from the farthest edge of the track, then, it assumes a position of adjusting the speed until it reaches this velocity while driving towards the sensor with the biggest read value. The distance of this method conveys the greater length the car may advance with little or sometimes without significant steering changes. This state also brakes, if necessary, in situations that the car assumes a speed higher than the one calculated to be the target.
	
\subsubsection{Exception Behavior}
	
	
	\emph{Out of Track} is when, for any reason, the car is found outside of the track limits. In this case, the proper behavior is to try to return to the lane. Since there are many ways that a car may be out of the track, reentering it in an efficient way might require possible different angles as well, so, every time the car exceeds the inside boundaries of the lane, a proper returning angle is calculated. In road tracks, the outside track normally has a different terrain, sometimes dirt-based, meaning that skidding frequently occurs, and in an effort to avoid this, a control system to brake when the car begins skidding above a threshold was implemented. Figure~\ref{Fig:Angle} demonstrates a car returning from outside of the track with a certain angle. 
	
	\begin{figure}[h]
			
			\centering
			\includegraphics[width=250pt]{ReturnAngle}
			\caption{Angle between car and track axis.}
			\label{Fig:Angle}
			
	\end{figure}
		
	Furthermore, the \emph{Stuck} state represents any given situation that the car is unable to progress in the race. This is a delicate state, because it presents itself as difficult to identify and also due to its impact to the performance of the controller. In order to detect \emph{Stuck} circumstances, the speed of the car is monitored throughout the race, during every game tick, if it lingers with a low speed for a determined period, then it is considered stranded, or stuck. This state activates the reverse gear of the car and turns it until its front is directed towards the correct axis of the track. The reason why \emph{Stuck} is a sensitive state is because, when detected early, might indicate false positive, and, when detected late, could lessen the efficiency of the controller. Thusly, detecting \textit{Stuck} situations is crucial, and so is handling the car out of them.
	
\subsection{Five-State FSM} \label{subsec:FSM5}

	The first conceived method was named ``Five-State FSM'', naturally because it is a finite state machine with five states. The real first model of the finite-state machine did not have an \emph{Approaching Curve} state, but, as the implementation of \emph{Straight Line} and \emph{Curve} were so different from one another, a preparation had to be established so as to smoothen the transitions between them. Figure~\ref{Fig:FSM5Diagram} represents the states diagram of the Five-State FSM.
	
	\begin{figure}[h]
		
		\centering
		\includegraphics[width=250pt]{FiveStateFSM}
		\caption{States diagram of the Five-State FSM}
		\label{Fig:FSM5Diagram}
		
	\end{figure}
	
	Figure~\ref{Fig:FSM5Diagram} instantiates the relations between the states through two-headed arrows connecting a state to every other one, which is done because, from a particular state, the controller can assume any other in a following instant.
	
	In order to decide wich state should be active at each moment a transition function was defined. It would take into account the covariance among the range finders. See Figure~\ref{Fig:FSensor}
	
	One big problem about this approach is that the function responsible for choosing which state is more appropriate for each situation would more than often be overcharged, and, in some cases, rather different sets of parameters received by it would result in the same classification among the states. Thus, in order to minimize the dependency of the driving performance in relation to the function in charge of the transition between states, a project decision was made to reduce the number of states.
	
	In addition, the angles - with relation to the car axis - of all the 19 range finder sensors were chosen in a manner that included as much information of the track as possible. For instance, if all the angles were to be initialized pointing only to one side of car, the data concerning the other side would be neglected. For the Three-State FSM, the angles were instantiated using steps of 10 degrees, which resulted in angles ranging from -90 to +90 degrees, 0 meaning the direction that indicates the front of the car.
		
\subsection{Three-State FSM} \label{subsec:FSM3}
	
	The finite state machine with less states was designed from a derivation of the previous one, by adapting the normal behavior and simplifying it to form the denominated Three-State FSM. The exception behavior remained the same, since it should not affect the car as much if the adaptation was well performed. Figure~\ref{Fig:FSM3Diagram} shows the modified states diagram of the Three-State FSM.
	
	\begin{figure}[h]
		
		\centering
		\includegraphics[width=230pt]{ThreeStateFSM}
		\caption{States diagram of the Three-State FSM}
		\label{Fig:FSM3Diagram}
		
	\end{figure}
	
	Just as the diagram of the Five-State FSM portrayed, there is no restriction of order concerning the states for the Three-State FSM. Figure~\ref{Fig:FSM3Diagram} then displays this relation, which is the possibility of assuming an arbitrary state after any other.
	
	The transition function in this architecture is simplier than the previous one. It only checks one range finder since it is very easy to say whether the car is inside or outside the track.
	
	Besides the states, there is also a learning module that is called whenever the \emph{Out of Track} state is requested. This module records both speed and position from the state of the car in the proximity of the departure from the track, and retaining these pieces of information allows the controller to slow down in subsequent laps when approaching the segments highlighted by the learning module. The implementation of this procedure consists on replacing the speed recorded from the landmarked position by a slower one on future occurrences.
	
	In order to maximize the efficiency of the information received from the track, the vector of sensors in the Three-State FSM was initialized according to a normal distribution, i.e., the sensors are more densely distributed in front of car and less on the sides.
	
	Each state in the two ways described to deal with the process becomes, ideally, an independent problem, whose solution can be attacked separately. This way, they can all have individual sets of parameters susceptible to improvement.
	
\subsection{Search for Parameter Values - Genetic Algorithm} \label{subsec:GA}
	
	Due to the quantity of parameters to be tuned and the defined granularity, the search space becomes enormous and renders fundamental the use of a search algorithm. This technique optimizes the process of finding better configurations by being more incisive and saving resources such as computational time and space. In the present study, an evolutionary algorithm was chosen for this task.
	
	Genetic Algorithms~\cite{GA} are evolutionary algorithms inspired by nature, in special by the concept of evolution through natural selection~\cite{Darwin}, whose main idea is that a set of solutions for a problem can be evolved like the population of a generic species in nature. The applications of genetics algorithms are present in many areas, extending from control engineering for non-linear systems identification~\cite{GACTRL} to biomedicine prosthesis development~\cite{GABIO} and even economy to forecast the behavior of agents~\cite{GAECO}.
	
	In this context, a solution is called an individual and was represented by a string of parameters. The first population was instantiated randomly to its full extent since there are no clues concerning how good an initial set of predefined parameters can be. This population passes through a fitness function that indexes a score to each individual, and this function is responsible for assessing how good - or how adapted - the solution that is being evaluated is. After evaluating the population, a group of individuals is chosen as the parents of the next set of solutions, which would compose a new generation; there are countless ways of performing the selection of the parents to the new generation of offspring, and this work gave preference to picking the higher individuals on the scoring system, what is called Elitism~\cite{ELITISM}. Each pair of parents was submitted to crossover in order to generate two offspring solutions, and in the end of the process each offspring might have presented mutation - everything according to predefined rates. The crossover took place in 95\% of the reproduction, while the mutation rate assumed the rate of 1\%~\cite{RATES}.
	
	For the Five-State FSM, 22 parameters required adjustment, which originated, in different quantities, from each state separately and also the \emph{transition} function, as follows:
	
	\begin{itemize}
		
		\item \emph{Transition Function} has 3 parameters;
		
		\item \emph{Approaching Curve} has 4 parameters;
		
		\item \emph{Straight Line} has 4 parameters;
		
		\item \emph{Out of Track} has 7 parameters;
		
		\item \emph{Stuck} has 4 parameters.
		
	\end{itemize}
	
	However, for the Three-State FSM, only 17 parameters demanded adjustment, which are divided as follows:
	
	\begin{itemize}
		
		\item \emph{Inside Track} has 6 parameters;
		
		\item \emph{Out of Track} has 7 parameters;
		
		\item \emph{Stuck} has 4 parameters.
		
	\end{itemize}
	
	 The source codes for both models presented in this section are available at the \emph{GitHub} repository provided in the references~\cite{GitHub}.

	%\section{Genetic Algorithm}
	Due to the quantity of parameters to be tuned and the defined granularity, the space search is enormous and it's necessary some space search algorithm. In this study was chosen a evolutionary algorithm.
	
	
	A genetic algorithms are adaptive heuristic search algorithm based on the evolutionary ideas of natural selection. Genetics algorithms applications are present at many areas such as control engineering at non-linear system identification, biomedics to develop legs mechanism to proteses and even economy to forecast the behavior of agents.  
	
	Each individual is a string that contain the information necessary to evaluation. Usually, the values that composes the strings are parameters to be tuned. The search space of the optimization of parameters is defined by the representation of the chromosome.
		
	
	A randomly initial population is created, than, accordingly with the fitness function, is associated a score for each individual that represents how good the individual is.	
	
	
	Based on the premise that the better ones is more likely to reproduce a better offspring than others, the parents are chosen by the score of each individual. Thus your genetic material can be spread.
	
	
	To increase a genetic diversity each chromosome has a tiny chance to be mutate, this avoid a fast convergence of the algorithm, trying not to be stuck in a maximum local.
	
	
	For the model with 5 states, it was necessary adjust 22 parameters, each state has a different quantity, including the \emph{transition} function, as follows:

	\emph{Approach Curve} has 4 parameters.

	\emph{Transition Function} has 3 parameters.

	\emph{Straight Line} has 4 parameters.

	\emph{Out of Track} has 7 parameters.

	\emph{Stuck} has 4 parameters.
	
	
	

	\section{Experimental Results} \label{sec:Experiments}

	A veil was put over the Five-state FSM from the very beginning of the experimentation phase, which was its great
	dependency towards the transition function. Early superficial assessments of the performance of this first model
	indicated an overcharge concerning this function, which was verified by considerable changes in behaviour derived
	from adjustments in its parameters. For that reason, a second model with less states - the Three-state FSM - was
	designed regarding this characteristic and releasing part of the performance burden from the transition function,
	and the results from the comparison of these architectures are described and analyzed in the succeeding
	subsections.

\subsection{Methodology} \label{subsec:Methodology}

	Once defined the models and which of their parameters required tuning, the genetic algorithm was applied to each
	one separately, in order to adjust their configurations for superior and competitive status. At the same time, the
	goal was to find general and versatile controllers with good results for any track and specific ones that were
	fittest to race in various tracks: road, dirt and oval alike. Therefore, due to the differences observed in the
	parameters of controllers evolved on dirt and road tracks, the evolution process took place separately for those
	two kinds of environments, which produced two contrasting sets of enhanced values for each model.
	
	The \emph{metric} chosen to evaluate the generated controllers was the combined sum of the distance raced by the
	car alone in the first 10.000 game cycles - also called \emph{game tics} - of a list of mixed tracks. This value
	will henceforth be called the \emph{fitness} of the controller, as it was used to determine whether he would
	remain in the evolution process.
	
	The experiments concerning Oval Tracks would repeatedly provide inconclusive results, so they were neglected in
	this evaluation process. Thus, in order to find the best set of parameters for a general track, the two
	finite-state states were evolved in three different sets of tracks, one with 4 Dirt Tracks, one with 4 Road
	Tracks and another with the 4 of each type. The evolution process for each set of tracks consisted on 600
	generations of 30 individuals and culminated in one controller; in other words, At the end of the experiments,
	there were six evolved controllers, one specific for Road Tracks for each model, one specific for Dirt Tracks for
	each model, and also one evolved in a mixed manner for each model.
	
	The validation occurred through testing the six produced pilots in a predefined set of tracks different from those
	in which they were evolved, lest they would evidence too track-limited parameters. On behalf of comparison, the
	results from the AUTOPIA controller were incorporated in the analysis, since it can be considered the State of
	the Art, displaying one of the best performances for the SCR Championship, and should provide satisfactory basis
	for appraisal.
	% the set of tracks was defined by autopia
	% tracks image as in the autopia's article
	% explain warm up stage
	
\subsection{Results} \label{subsec:Results}

	% is the fitness graphic among generations appropriate here?
	% add tables with the evaluation results
	
\subsection{Analysis} \label{subsec:Analysis}
	
	
	
	\section{Conclusions} \label{sec:Conclusions}

	This paper proposed two approaches developed to control a car during a race in a simulated computational environment, the game platform TORCS. The models of both of these controllers were described, explained, enhanced by means of a genetic algorithm, compared and then tested together with a State of the Art controller - AUTOPIA. It was implied before the testing phase that a finite state machine too burdened in the process of transition between states might lose performance, which was corroborated by the experimental results of the comparison of the two models detailed.
		
\subsection{Conclusions} \label{subsec:Conclusions}

	The interpretation of the global results from the experiments performed gives margin to declare that finite state machines are a reliable technique to implement artificial intelligences, at least for computer games such as simulated races. They provide the possibility of parallel development and also enable parameter tuning in separate fronts, due to the independence and abstraction between the behaviors from each state.

	The finite state machine with less states achieved a superior overall performance in the tests carried out. The simplifications fashioned in its transition function were inferred to be the reason for this improvement, along with its intricate relation with the number of parameters that were target of fine tuning in the evaluation and validation process.
	
	The evolution procedure adopted concerning the controllers culminated in three characteristic behaviors. The controller evolved on road tracks became a fast driver, whose hastiness resulted in a careless attitude in general; in other words, it was only good for races in road tracks. The one evolved on dirt tracks turned out to be too careful in contrast, limitedly determining its speeds and, in efficiency terms, inferior. The controller evolved on a mixed set of tracks inherited characteristics from both the previous ones, becoming swift but not too hasty, prudent but not too slow. The latter surpassed the performance of the dirt-evolved drivers even on dirt tracks, and did not lose by far on road tracks in comparison to drivers evolved solely in them.

\subsection{Future Works} \label{subsec:Future}
	
	\toDo{Falar melhor dos trabalhos futuros.}
	\toDo{Since at the SCRC the controller is allowed to perform a warm-up before the race, it is possible to acquire the track data, not only mapping critical section such as sharp curves and points where the car go out the track to improve the result of the controller at the race itself. Also a warp-up stage would supply the information about environment where the car is, including the type of track,road or dirt, which determine a set of parameters best fitted to which occasion. (Já foi feito.)}
		
	One important task to be accomplished is the opponent treatment in real-time races. Routines to reduce collisions, i.e., avoiding being overtaken and also being concerned about overtaking the opponents is a fundamental issue. Ignoring adjacent cars usually causes the driver to face unexpected collisions, ending up stuck, considering a worst case scenario. Many of the renowned developers for TORCS already incorporate such treatment in their controllers, and neglecting this necessity renders any driver less robust to unexpected race events, and also reduces its performance.
	
	
	%\section{Future Works} 	
	\begin{itemize}
		 \item Since at the SCRC the controller is allowed to perform a warm-up before the race it is possible to acquire the track data, not only mapping critical section such as sharp curves and points where the car go out the track to improve the result of the controller at the race itself. Also a warp-up stage would supply the information about environment where the car is, including the type of track,road or dirt, which determine a set of parameters best fitted to which occasion. 
        
		\item One important task to be accomplish is the opponent treatment, routines to reduce collisions, avoid the controller to be surpassed and surpass the opponents is a fundamental issue. Ignoring the opponent would make the driver to face unexpected collisions ending stuck, considering the worst event.
	\end{itemize}
	

	


	%\section*{Acknowledgements}

	\bibliographystyle{sbgames}
	\bibliography{Bibliography}

\end{document}
