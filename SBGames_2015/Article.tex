\documentclass[a4paper]{sbgames}
%\usepackage[scaled=.92]{helvet}
\usepackage{times}
\usepackage{graphicx}

%% use this for zero \parindent and non-zero \parskip, intelligently.
\usepackage{parskip}

%% the 'caption' package provides a nicer-looking replacement
\usepackage[labelfont=bf,textfont=it]{caption}

\usepackage{url}
\usepackage{algorithmic}%
\usepackage{algorithm}%
\usepackage{multicol}%
\usepackage{multirow}%
\usepackage{xcolor}%
\usepackage{subcaption}%
% \usepackage[section]{placeins}%
 \usepackage{placeins}%

\newcommand{\toDo}[1]{\textcolor{red}{#1}}

\newcommand{\insertPicture}[1]{\textcolor{green}{#1}:
	\begin{figure}[h]
		\centering
		\includegraphics[width=150pt]{defaultPicture.jpg}
		\caption{\label{Label}Proper picture.}
	\end{figure}
	\\
	}

\title{Evolving Finite-State Machines Controllers for the Simulated Car Racing Championship}

% \author{Manuscript number $<$147281$>$}
% \contactinfo{147281}

\author{Bruno H. F. Macedo\\Gabriel F. P. Araujo\\Gabriel S. Silva\\\and Matheus C. Crestani\\Yuri B. Galli\\Guilherme N. Ramos\\\hspace{-12em}\textit{University of Bras\'{i}lia}}

\contactinfo{gnramos@unb.br}

\keywords{finite-state machine, genetic algorithm, TORCS, simulated car racing}
\newcommand{\gnramos}[1]{\textcolor{red}{#1}}%
\begin{document}

	\maketitle

	\begin{abstract}
		Autonomous vehicles have many practical applications, but the development of software controllers for such use has several difficulties. This work presents a finite state-machine model with evolved parameters as a suitable solution for a self-driving car, an approach that enables a clear division of behaviors in states, providing an easy way to test different configurations and simplifying the search for better controllers by allowing changes only in selected states. A 5-state and a 3-state drivers were evolved through genetic algorithm, and a learning module developed to improve their behaviors. These were compared to each other and to AUTOPIA, the current state of the art controller for the Simulated Car Racing Championship, using The Open Racing Car Simulator. Results showed that the proposed model has potential for racing, besting one of AUTOPIA's qualifying marks, and provide insights on developing and configuring the model.
	\end{abstract}

	\keywordlist
	\contactlist

	\section{Introduction}\label{sec:1}

Automation of day to day tasks is an endeavor that has moved a large amount of scientific resources in the recent history~\cite{INDUS,APPLI}. One of the more desirable goals is the advent of self-driving cars, which should drive safely and efficiently, within traffic laws~\cite{SAFE,AUTOM}, bringing hope to current issues of traffic in urban roads by aiming at shorter response times, better fuel consumption, and lower levels of pollution. Such drivers, however, face several difficult challenges such as perception, navigation and control~\cite{6179503}.

Another critical issue is driver development, since testing is a frequent task and real driving systems are expensive. Creating and testing solutions can be aided by realistic car racing games which can closely simulate the roads, other vehicles, and their complex interactions~\cite{caldeira2013torcs}. Games also present a well defined environment which may be used not only for applications of machine learning results, such as neuroevolution~\cite{stanley_real-time_2005,5482132} or human pose recognition~\cite{Shotton:2011}, but also for comparing AI solutions for specific problems such as path planning~\cite{deFreitas:2012}, controlling human non-playable character~\cite{simon2008}, car racing~\cite{2009}, among others.

The Open Racing Car Simulator (TORCS) is a modern, modular, highly-portable multi-player, multi-agent car simulator~\cite{SIMUTORCS}, one of the most advanced racing games available, and frequently used as a platform for comparing driver solutions in the Simulated Car Racing Championship (SCR)~\cite{2009,Loiacono:2012:LEA:2212908.2212953}. This paper uses a finite-state machine (FSM) controller to exploit the advantages of a divide-and-conquer approach by considering the task's various situations as distinct states (racing, getting unstuck, getting back on track, etc.).

FSM is a well-known mathematical model with widespread applications such as controlling air conditioning systems~\cite{BERNARD}, highway surveillance systems~\cite{DOHYUN}, and even simulated car racing~\cite{DIEGO}. Such solutions are implemented as software and configured according to a specified set of parameters, and defining the best possible configuration is usually a huge combinatorial problem for which exhaustive systematic searches become unfeasible. Thus, several machine learning approaches have been applied to it (heuristic algorithms~\cite{MrRacer}, evolutionary algorithms~\cite{Nallaperuma:2014}, modular fuzzy architectures~\cite{AUTOPIA}, sensory-to-motor couplings~\cite{COBOSTAR}, among others). One of the most successful optimization methods is Genetic Algorithm (GA), which is has been applied to a myriad of problems such as non-linear systems identification~\cite{GACTRL}, biomedicine prosthesis development~\cite{GABIO}, agent behavior forecasting~\cite{GAECO}, improving classifiers~\cite{pedrycz_genetic_2005}, and others.

This work uses a GA for searching for optimal parameter settings for the proposed FSM based driver, using TORCS as test bed. The rest of this paper is structured as follows: Section~\ref{sec:2} introduces the TORCS/SCR, finite-state machines, and Genetic Algorithms concepts involved; Section~\ref{sec:3} details the proposed controller model. Section~\ref{sec:4} describes the evaluation and validation process; and Section~\ref{sec:5} presents concluding remarks.
%
	\section{Background}\label{sec:2}

Scientific Research has been greatly aided by the evolution of simulators, which greatly simplify and reduce initial development costs. These tools have been widely used in several areas, such as medical education~\cite{MEDIC}, aiding decision making~\cite{useOfSimulaton2002}, aviation industry~\cite{AIR}, and automotive research and development~\cite{AUTR}. In the field of Machine Learning, simulations are specially helpful for training and learning through experimentation.

Car simulators model several elements of a vehicular dynamics, including inertia, suspension types, differentials, friction, aerodynamics, and others~\cite{SIMUTORCS}. These models represent an approximation of real systems, and the reality gap (differences between model and real system results~\cite{brookes2012authentic}) that stems from the simplifications made have to be considered, so simulations cannot completely replace experimentation on actual cars. However, current advanced simulators provide such realistic experiences that they are ideal for most of the development phase.

\subsection{TORCS \& SCR}
The Open Racing Car Simulator\footnote{\url{http://torcs.sourceforge.net/}} is a platform that is renowned for its highly credible physics modeling engine and yet user-friendly interface with very customizable environment for car racing simulations~\cite{SIMUTORCS,SCR}, and has been widely used in Artificial Intelligence (AI) for developing and comparing solutions~\cite{2009}. The engine of this simulator considers factors such as collision, traction, aerodynamics, and fuel consumption and provides several circuits, vehicles, and controllers~\cite{2009,Loiacono:2012:LEA:2212908.2212953}, enabling all kinds of possible in-game situations. Additionally, it is open source, with an active community, making it possible and encouraging modifications of its source code to better suit specific needs. It has been used as a standard platform for simulated racing since 2007~\cite{Loiacono:2012:LEA:2212908.2212953}.

% Mass, rotational inertia of the car, engine, wheels, and other components, are included in the model of the vehicular system; while the types of different suspension, links, and differentials are done so in the mechanical model. The profiles for different ground types with both dynamic and static friction are also included; this way, the aerodynamics modeling includes slipstreaming and ground effects, that vary from one profile to another. Nevertheless, the simulation engine can be replaced or easily modified as a result of the modularity supplied by TORCS. The interface with this platform occurs by means of a sensor-based interaction system in which the developer is able to interpret received parameters of the car - such as speed in X, Y and even Z axes - and control the car through programming its actuators, some of which are acceleration and steering.

The TORCS software\footnote{\url{http://arxiv.org/pdf/1304.1672.pdf}} represents a stand-alone application in which pieces of program code may be used to drive the simulated car robots, which represent every in-game opponent. The specific robots that are loaded and compiled with the game as artificial intelligences are denominated ``bots'' inside the environment, and, consequently, as there is no separation layer between them and the simulation engine, they retain full access to the information concerning the structure and the status of the race. There is a great diversity concerning types of cars and tracks available at TORCS, and also around 100 bots to race against. The tracks differ in categorization according to their variety of ground, which affects the dynamics of the cars in execution time. The cars feature particular characteristics of tire and wheel properties, including aerodynamics and others. The players may also play the part of developers by coding their own robots to the game, which become what are called ``controllers''.

% Race tracks are categorized into \emph{Road}, \emph{Dirt} and \emph{Oval},
% and several types of cars, such as
% TORCS allows the controller to have a full view of the environment including its exact location inside the track, the geometry and friction and also the exact location of all the other cars.


% In TORCS, the participating players are referred to as ``robots''. They are loaded as external modules in TORCS. This means that new artificially intelligent agents can be developed independently and they only have to satisfy the basic API requirements for robot code. At the moment, a large number of dedicated TORCS robots exist, some of which can operate at a level exceeding that of human performance in the game. Consequently, they form a challenging metric against which any new AI player can be evaluated ~\cite{SIMUTORCS}.


The Simulated Car Racing Championship (SCR) is a competition between controllers built on TORCS~\cite{SCR}. It was the first simulated car racing championship organized as a joined event of major scientific conferences: IEEE Congress on Evolutionary Computation, the ACM Genetic and Evolutionary Computation Conference, and the IEEE Symposium on Computational Intelligence and Games~\cite{2009}, and has been accepted by researchers as suitable platform for evaluation and comparison of controllers~\cite{SIMUTORCS}.

Controllers are ranked according to performance during the championship, which consists of several races on different tracks divided into legs, spread through the conferences~\cite{2009}. They are scored using the Formula 1 point system\footnote{\url{http://www.formula1.com/}}. The software for SCR extends the TORCS architecture by structuring it as a client–server application, by incorporating real time processing and by physically separating the driver code and the race server through a sensors/actuators model abstraction layer~\cite{2009}. These changes provide an even more interesting environment for researchers by enabling any kind of controller implementation (as long as it can communicate via UDP connections) and defining a clear interface between controller and simulated car, which can be easily adapted for testing the controller with a different simulator or even a real car (provided the proper adaptations).

%	\begin{figure}[h]

%	\centering
%	\includegraphics[width=250pt]{Figure1.jpg}
%	\caption{Available data inside TORCS becoming data accessible to the client}
%	\label{Fig:Comm}

% \end{figure}

% Race tracks are categorized into \emph{Road}, \emph{Dirt} and \emph{Oval} inside TORCS. The races from the SCRC take place in track types decided by the organization of the championship, information which is not provided to the participants and that may incorporate maps that are unknown to them. The competition adopts a structure that gathers a \textit{Warm-up} stage, a \textit{Qualifier} stage and a \textit{Final} race. Noise can be introduced in the sensors, option that is present during the actual competition. The complete sensorial input information and all the details concerning the race stages and types are presented at the Simulated Car Racing Championship Competition Software Manual~\cite{SCRC}.

% The reason why TORCS presents itself as a satisfactory AI benchmark, in combination with SCR, is because even	though there are multiple possibilities on how the sensorial input received from the server can be translated into the behavior of the actuators, they can all be compared in a race, which has a robust and steady scoring and evaluational system. In other words, there are many different approaches concerning how to teach the racer encoded by the developers to drive in a racing competition only with the information given by the sensors, and the metric to that issue is the performance on the race itself.



% \section{\textbf{Related Works and State of the Art}} \label{sec:Related}

SCR controllers have been developed using AI techniques, such as neural networks, fuzzy logic, potential fields, and genetic algorithms~\cite{Loiacono:2012:LEA:2212908.2212953}. In the competition, the race track is initially unknown and it several drivers incorporate machine learning procedures to improve their performance~\cite{2009} advantage, which can essentially be \emph{online} or \emph{offline}. Online systems learn by moving about the environment and observing the results while offline ones learn solely by simulating actions within an internal model~\cite{mitchell_1997}.

\emph{Mr. Racer}, which has won the last three championships (2011 to 2013), employs several heuristics and black-box optimization methods in a a modular structure in order to reproduce human-like mechanisms~\cite{MrRacer}, it applies a Covariance Matrix Adaptation Evolution Strategy (CMA-ES), to evolve parameters offline.

In the GECCO leg of the 2013 competition\footnote{\url{http://www.slideshare.net/dloiacono/gecco13scr}}, \emph{AUTOPIA}'s performance stands out. It implements a fuzzy architecture with gear, steering and speed control modules which are optimized offline by a genetic algorithm and an online learning mechanism for landmarking lane exit points to avoid leaving the track~\cite{AUTOPIA}.

Not surprisingly, other participants also use combinations of offline and online learning and modular approaches~\cite{2009,DIEGO,Exp}. Modularity has the clear advantage of independent development and optimization\cite{MrRacer,AUTOPIA2009}, and one of the simplest models for implementing different behaviors is a finite-state machine.

\subsection{Finite-State Machines}%
A FSM is a mathematical model with a finite number of states that can transition from one to another, and a FSM whose output values are determined solely by its current state is a Moore machine~\cite{Ajzerman}. FSMs have been widely used in AI application~\cite{Millington:2006:FSM}, mostly due to its inherent characteristics of flexibility, modularity, and intuitive behavior, among others~\cite{Buckland:2005:AI}. Since the FSM is a widely known model to handle robot engines problems it seems to be highly applicable to the controller. 

States are usually implemented with hard-coded rules concerning a specific situation~\cite{Buckland:2005:AI}, which in turn demands small amounts of processor time, and can be easily implemented in different manners. The different driving behaviors can then be coded in parallel with less effort, and easily translated into different states of a FSM. Such approach has successfully been used in SCR~\cite{2009,DIEGO}.

The proper configuration of states and transitions, however, are a more complex problem. Several possible solutions exist, and machine learning techniques can readily be applied.

\subsection{Genetic Algorithms}

GAs are a particular kind of genetic optimization mechanism, inspired by evolutionary algorithms inspired by Darwin's theory of natural selection~\cite{GA}. They are probabilistic search procedures designed to work on large spaces~\cite{goldberg1988}, and have been successful applications in many areas~\cite{GABIO,GAECO,stanley_real-time_2005,pedrycz_genetic_2005}.

GAs work by generating successor solutions by repeatedly mutating and recombining parts of the best currently known solutions, replacing a fraction of the population by offspring~\cite{mitchell_1997}. This is interesting because it works with little information on the problem's domain, the only requirements related specifically to the problem are a representation of a solution for a problem and a function for evaluating it's quality (fitness).

In the context of a self-driving car controller, a solution can be seen as a representation of the driver's parameters and the fitness a measure of its performance, considering speed, safety, fuel consumption, or whatever is the developer's interest. Considering SCR, the usual fitness is the distance driven during the given simulation time~\cite{2009}.
%
	\section{FSMDriver}\label{sec:3}%
Driving a car can be intuitively divided into several behaviors, and thus provide a straightforward implementation as a finite-state machine model. Such model's configuration parameters can be optimized through a genetic algorithm, and the TORCS/SCR platform provides a standard measure of driver quality, and its sensor/actuator interface enable it to be readily replaced by actual robotic prototypes (or other more advanced models) for further testing. This work proposes to use these ideas for evolving a controller for a car, the FSMDriver, as one more step towards autonomous vehicles.

\newcommand{\state}[1]{\texttt{#1}}%
\newcommand{\SL}{\state{Straight Line}}%
\newcommand{\C}{\state{Curve}}%
\newcommand{\OT}{\state{Out of Track}}%
\newcommand{\St}{\state{Stuck}}%
\newcommand{\AC}{\state{Approaching Curve}}%
\newcommand{\IT}{\state{Inside Track}}%

\subsection{Driving Behaviors}%
A \state{State} in the FSM model considered here defines a racing behavior, deciding the driver's output according to the given sensor input as defined by SCR's API. The FSM implements a transition function that, at every game tick, analyses the input and decides which state is appropriate, triggering a change if necessary. The state then processes the input and defines the proper output.

The next step for implementing the FSM model is defining its states. Intuitively, \state{Driving}~could be a definite solution but it obviously involves several distinct situations which should be divided, simplifying the development.

Considering the testing environment, it is clear that two antagonistic situations that require specific behaviors: \state{Racing}, for situations where the car is within track limits, facing the right direction; and \state{Recovery}, for when the car is outside track limits or facing the right direction, or unable to move.

These can be further divided into two more specific behaviors for implementation:
\begin{description}
	\item[\SL:] for racing straight ahead;
	\item[\C:] for racing through a curve;
	\item[\OT:] for recovering from leaving the track or facing the wrong direction; and
	\item[\St:] for recovering from being unable to move forward.
\end{description}

Each of these states implements its behavior as follows. \SL~ attempts to go as fast as possible parallel to the track axis by accelerating at full throttle, and changing gears according to RPM thresholds. \C~steers towards the direction towards the track sensor with the largest reading (see Figure~\ref{Fig:FSensor}) with 60\% throttle, braking in case it starts to slide in the X axis. \OT~ attempts to return to the track, facing the right direction, adjusting its speed and steering according to its current orientation concerning the track (see Figure~\ref{Fig:Angle}). \St~ tries to get unstuck by using the reverse gear and hard steering.

\begin{figure}[h]
	\centering
	\includegraphics[width=.45\textwidth]{img/FarthestSensor}
	\caption{Sensor input in curve.}
	\label{Fig:FSensor}
\end{figure}

\begin{figure}%[h]
		\centering
		\includegraphics[width=.45\textwidth]{img/ReturnAngle}
		\caption{Angle between car and track axis.}
		\label{Fig:Angle}
\end{figure}

\subsection{FSMDriver5}

Initial tests showed that transitioning from \SL~to \C~was too swift, hampering performance because the car consistently entered curves at such high speeds that the controller was unable to turn and exited the track, so the additional behavior \AC~was implemented. \AC~ positions the driver towards the outside of the incoming curve and tries to race at a speed proportional to the curvature, so less braking would be necessary in \C. Thus, a 5-state FSMDriver (FSMDriver5) model (illustrated in Figure~\ref{Fig:FSM5Diagram}) was ready for testing.

\begin{figure}[h]
	\centering
	\includegraphics[width=.45\textwidth]{img/FiveStateFSM}
	\caption{FSMDriver5 state diagram.}
	\label{Fig:FSM5Diagram}
\end{figure}

The range sensors were arbitrarily configured at every $10$ degrees from $-90$ to $90$ relative to the car's front axis (that is, a uniform distribution from the left to the right of the car), as illustrated in Figure~\ref{Fig:FSM5Sensors}.

\begin{figure}[h]
	\centering
	\includegraphics[width=.45\textwidth]{img/FSM5Sensors}
	\caption{FSMDriver5 range finders.}
	\label{Fig:FSM5Sensors}
\end{figure}

Sensor readings provide an idea of location in the track, in this setting, the difference between frontal and lateral sensors is larger when driving in a straight line than in a curve. Thus, the transition function considers the variance of readings from range finders to define the current state of the \state{Racing} behavior, as described in Algorithm~\ref{alg:FSMDriver5}.

\begin{algorithm}[h]%
\caption{FSMDriver5 Transition}%
\label{alg:FSMDriver5}%
\begin{algorithmic}
    \IF{car is stuck}
        \STATE $state \Leftarrow Stuck$
    \ELSE
	\IF{($variance < MAX\_STRAIGHTLINE\_VAR$) \OR ($variance < MIN\_STRAIGHTLINE\_VAR$ \AND
   	the current state is not $StraightLine$)}
   		\STATE $state \Leftarrow StraightLine$
	\ENDIF
	\ELSIF{($variance >MAX\_APPROACHING\_CURVE\_VAR$ \AND current state is not curve)
         \OR ($variance > MIN\_APPROACHING\_CURVE\_VAR$ \AND current state is not Approaching Curve)}
		\STATE $state \Leftarrow Approaching Curve$
	\ELSIF{$variance>0$}
		\STATE $state \Leftarrow Curve$
	\ELSE
		\STATE $state \Leftarrow Out Of Track$
	\ENDIF
\end{algorithmic}
\end{algorithm}

\gnramos{mudar MAX\_STRAIGHTLINE\_VAR para MAX\_SL\_VAR e MAX\_APPROACHING\_CURVE\_VAR para MAX\_AC\_VAR}%

% In order to decide which state should be active at each moment a transition function was defined. It would take into account the covariance among the range finders. See Figure~\ref{Fig:FSensor}.

%The Five-State FSM presents a very complex transition function that takes into account the variance of the sensorial input to decide if the car is in a straight line, approaching a turn or in a turn

Initially, random configurations were set and manually adjusted until an acceptable behavior was achieved, i.e. such a configuration in which a car could successfully complete a race. In the beginning, the \state{Recovery} states (\OT~and \St) were constantly triggered, and therefore were first to be adjusted for improved behavior. Eventually their settings were such that the \state{Racing} behaviors could be focused on. These too were manually and arbitrarily set through trial and error, considering the driver's quantitative performance racing (time and damage) and its qualitative skill (visual analysis of tests).

After several iterations, this model performed better performance than some of the less capable robots available in TORCS,showcasing the FSM model's potential for controlling a car. However, testing also revealed that the transitions between states required were the bottleneck for improved racing. Not only some of the triggers needed a more detailed analysis to avoid eventual erratic behavior, but the sheer number parameters considered for transitioning, each with its triggers (see Figure~\ref{Fig:FSM5Diagram}), and their impact on the model's behavior during a complete race caused the whole model to be thought over.

Analyzing FSMDriver's transitions occurrences within a race, it was clear that the \state{Recovery} behavior had two distinct situations, which were acceptably handled by the current configuration, and that the major issue was frequent triggering between the \state{Racing} states, which inevitably resulted in the car leaving the track and jeopardizing its performance.

\subsection{FSMDriver3}%
To reduce the model's complexity, and considering the intuitive division of behaviors proposed, a new model where the defined \state{Recovery} behaviors were maintained and the \state{Racing} behaviors were joined into a new state called \IT. Figure~\ref{Fig:FSM3Diagram} illustrates this FSMDriver3 model.

\begin{figure}[h]
	\centering
	\includegraphics[width=.4\textwidth]{img/ThreeStateFSM}
	\caption{FSMDriver3 state diagram.}
	\label{Fig:FSM3Diagram}
\end{figure}

\IT~implements a very straightforward idea, get to an arbitrary target speed towards the largest open space inside track limits. The target speed is proportional to the largest sensor reading, and braking is activated if the current speed is greater than the target. Gear changing works based on RPM threshold as before. Overall, this implementation works for straight lines as well as curves.

In order to have more detailed information on track segment straight ahead, which will guide the movement, the range finders are concentrated in front of the car, as shown in Figure~\ref{Fig:FSM3Sensors}, contrasting with FSMDriver5's equidistant distribution.

\begin{figure}[h]
	\centering
	\includegraphics[width=.45\textwidth]{img/FSM3Sensors}
	\caption{FSMDriver3 range finders.}
	\label{Fig:FSM3Sensors}
\end{figure}

The transition function can then be reimplemented in a simpler fashion, as shown in Algorithm~\ref{alg:FSMDriver3}:

\begin{algorithm}[h]%
\caption{FSMDriver3 Transition}%
\label{alg:FSMDriver3}%
\begin{algorithmic}
    \IF{car is stuck}
        \STATE $state \Leftarrow Stuck$
    \ELSE
        \IF{car is within track limits}
            \STATE $state \Leftarrow Inside Track$
        \ELSE
            \STATE $state \Leftarrow Outside Track$
        \ENDIF
    \ENDIF
    \IF{the current state is not $state$}
        \STATE $Current State \Leftarrow state$
    \ENDIF
\end{algorithmic}
\end{algorithm}

Initially, configurations resembling FSMDriver5's parameters were set and manually adjusted until a behavior similar to the original driver was achieved. As expected, this process was quicker than for FSMDriver5 and soon a configuration better than some robots was available.

With the finished models\footnote{Code available at \url{https://github.com/bruno147/fsmdriver}}, the issue at hand became to set the models' parameter to such values as to maximize their performance, a large combinatorial problem to which a well known tool was applied.

\subsection{Offline Learning: Evolving Parameters}%
In order to apply a genetic algorithm to the task of evolving the drivers' parameters, a model of the solution must be defined. In this context, a solution is called an individual and was represented by a string of bits, which compose the floating point parameters of the models.

FSMDriver5 has, overall, 22 parameters from states and the transition function:

\gnramos{Completar dados abaixo, e também para o FSMDriver3}

\state{Transition}:%
\begin{description}
	\item[MAX\_STRAIGHTLINE\_VAR] upper boundary for classifying a straight line.
	\item[MIN\_STRAIGHTLINE\_VAR] lower boundary for classifying a straight line.
	\item[MAX\_APPROACHIN\_VAR] upper boundary for classifying an approaching curve.
	\item[MIN\_APPROACHIN\_VAR] lower boundary for classifying an approaching curve.
\end{description}

\AC:%
\begin{description}
	\item[TARGET\_POS] desired percentage position.
	\item[BASE\_SPEED] lowest speed allowed during turns.
	\item[MAX\_STEERING] maximum allowed steering value.
\end{description}

\SL:%
\begin{description}
	\item[LOW\_GEAR\_LIMIT] threshlod to bound low gears.
	\item[LOW\_RPM] threshlod of rpm to delimit the change of low gears.
	\item[AVERAGE\_RPM] threshlod to decrease high gears
	\item[HIGH\_RPM] threshlod to delimit the change of high gears.
\end{description}

\OT:%
\begin{description}
	\item[MAX\_SKIDDING] defines the threshold to start to break to avoid skidding.
	\item[NEGATIVE\_ACCEL\_PERCENT] dictate how much to release the acceleration pedal to avoid to skidding, its defined with the axis speed of car.
	\item[VELOCITY\_GEAR\_4] threshold to change the gear to 4.
	\item[VELOCITY\_GEAR\_3] threshold to change the gear to 3.
	\item[VELOCITY\_GEAR\_2] threshold to change the gear to 2.
	\item[MAX\_RETURN\_ANGLE] upper boundary to angle of return to track.
	\item[MIN\_RETURN\_ANGLE] lower boundary to angle of return to track.
\end{description}
\St:%
\begin{description}
	\item[STUCK\_SPEED] defines the threshold to monitorate if its in stuck.
	\item[MINIMUM\_DISTANCE\_RACED] just to avoid to enter in stuck at the beginning of race when the car is stopped.
	\item[MAXIMUM\_NUMBER\_OF\_TICKS\_STUCK] maximum number of ticks in which reverse gear is allowed.
	\item[MAXIMUM\_NUMBER\_OF\_TICKS\_IN\_SLOW\_SPEED] maximum number of ticks in low speed before stuck is triggered.
\end{description}

FSMDriver3, on the other hand, has the same 11 for \OT~and \St~ as FSMDriver5, and 6 additional:

\IT:%
\begin{description}
	\item[LOW\_GEAR\_LIMIT] threshlod to bound low gears.
	\item[LOW\_RPM] threshlod of rpm to delimit the change of low gears.
	\item[AVERAGE\_RPM] threshlod to decrease high gears
	\item[HIGH\_RPM] threshlod to increase high gears.
	\item[SPEED\_FACTOR] proportionality between highest value read by range finders and TARGET\_SPEED.
	\item[BASE\_SPEED] lowest speed allowed.
\end{description}

The GA also requires a fitness function to evaluate the solutions, this work will consider the distance raced, which is also the standard metric for SCR. This means that the GA has to communicate with the test platform, calling  TORCS and simulating races with the drivers, and processing the results. The simulation is run through operating system calls and the results communicated through shared memory\footnote{Code available at \url{https://github.com/bruno147/driver-ga}}.
\gnramos{+ Informacao sobre a metrica}

\subsection{Online Learning: Staying in Track}%

An online learning module was included to improve performance, in an attempt to reduce damage and time loss for racing out of the track. Whenever the driver enters \OT, the speed and location of this occurrence are recorded and used in the \state{Racing} states. This information is used to slow down the driver in subsequent laps when approaching the recorded places, trying to remain inside the track.

\gnramos{+ Informacao sobre o modulo de aprendizado}%
	\section{Experimental Results}\label{sec:4}%
The drivers were developed for testing in TORCS\footnote{\url{http://sourceforge.net/projects/torcs/}} (v1.3.6) and SCR\footnote{\url{http://sourceforge.net/projects/cig/files/SCR\%20Championship/Server\%20Linux/}} (v2.1), since it provides an advanced simulation environment which provides complex situations, and enables objective metrics for direct comparison to exiting results. The standard the car model for racing is \emph{car1-trb1}~\cite{SCR}.

Once the FSMDriver models had been defined, a genetic algorithm was applied to each, aiming to improve their performances by changing their control parameters' values. During this stage, the learning module was disabled in FSMDriver3 in order to not interfere with the learning. The module is enabled for the next stage where the drivers' performance is evaluated by racing them in selected tracks, emulating the qualifying stage of SCR. Also as per the competition, before each qualifying race a warm-up stage is held in order for the controllers to adapt themselves to the track. This lasts for 5 laps, conforming to the stipulated 2015 championship rules\footnote{\url{http://cs.adelaide.edu.au/~optlog/SCR2015/}}.

Since it is unknown which tracks will be used during the SCR Championship, a more general and versatile controller is desirable and, thus, the controllers were evolved considering their performance in several tracks types. Three basic driver profiles were desired: a \emph{road driver}, which would perform well in road tracks, a \emph{dirt driver} for dirt roads and a \emph{mixed driver}, which should perform well in both types.

The Genetic Algorithm used standard configurations from literature, the population size is arbitrarily set as 30~\cite{RATES}, which leads to 616 generations for moderate problem complexity the optimal population size for problems coded as bitstrings~\cite{218485}.\gnramos{qual das formulas?} Selection is done through elitism by keeping the best 4 individuals from a generation unchanged, and the other 26 are generated from the best 10 individuals' reproduction~\cite{ELITISM}. This step is done by randomly choosing a pair from the top 10 that are cloned with a 5\% chance or generate a pair of offspring with a 95\% probability~\cite{RATES}. These offspring are produced through crossover by partitioning the parents' bitstrings at a random position and swapping them. Finally, mutation (flipping) is applied to every bit in every individual of these 26 with a 1\% probability~\cite{RATES}.

For the \emph{road profile}, four road tracks were selected from the standard TORCS distribution list to compose the training set in the evolution process. The tracks were arbitrarily selected for certain features, which might lead to better drivers: \emph{Spring}, the longest track available; \emph{Wheel 2}, with many sharp curves; \emph{E-Track 3}, a fast track with sharp curves; and \emph{Forza}, a fast circuit with smoother curves.

Analogously, four dirt tracks were selected for evolving a \emph{dirt profile}: \emph{Dirt 2}, \emph{Dirt 6}, \emph{Mixed 1}, and \emph{Mixed 2}, which present variations of high speed and curve styles. For the \emph{mixed profile}, \gnramos{quais 4 pistas?}

\gnramos{inserir exemplo de resultados de evolucao, talvez uma tabela mostrando os tempos em relacao a bots}

The best solutions evolved for each profile where then evaluated in a set of six tracks, different from the ones used before, which present distinct typed, speeds, and curve features for better analysis. These tracks, also available on the standard TORCS set, were used in previous SCR Championships~\cite{AUTOPIA2009}. They are three road tracks: \emph{Street 1}, \emph{D-Speedway}, and \emph{CG Speedway 1}; and three dirt tracks: \emph{Dirt 1}, \emph{Dirt 3}, and \emph{Dirt 4}. During this stage, the learning module was enabled.

% \begin{figure}
% \begin{subfigure}[b]{0.2\textwidth}
%        \includegraphics[width=\textwidth]{img/tracks/CG-Speedway1}
%        \caption{CG Speedway 1}
%        \label{fig:mouse}
%    \end{subfigure}
% \begin{subfigure}[b]{0.2\textwidth}
%        \includegraphics[width=\textwidth]{img/tracks/Street1}
%        \caption{Street1}
%        \label{fig:mouse}
%    \end{subfigure}
% \begin{subfigure}[b]{0.2\textwidth}
%        \includegraphics[width=\textwidth]{img/tracks/D-Speedway}
%        \caption{D-Speedway}
%        \label{fig:mouse}
%    \end{subfigure}
% % \begin{subfigure}[b]{0.2\textwidth}
% %        \includegraphics[width=\textwidth]{img/tracks/Mixed2}
% %        \caption{Mixed2}
% %        \label{fig:mouse}
% %    \end{subfigure}
%    \caption{Pictures of animals}\label{fig:animals}

% \end{figure}

\subsection{Single Race Results}

\gnramos{fsmdriver 5 na CG é isso mesmo?}
\begin{table*}
% \renewcommand{\arraystretch}{1.3}
\caption{Distance covered in meters racing alone for 10 000 game ticks}\label{tbl:ticks}
\centering
\begin{tabular}{| c | c | c | c | c | c | c | c |}
\hline
\multicolumn{2}{| c |}{\bfseries Driver} & \bfseries Street 1 & \bfseries D-speedway & \bfseries CG-speedway & \bfseries Dirt 1 & \bfseries Dirt 3 & \bfseries Dirt 4 \\\hline
\multirow{3}{*}{FSMDriver5}
& road & 3822.76 & 3427.11 & 4114.66             	& 2145.49 &        	2205.97 & 3260.19 \\\cline{2-8}
& dirt & 1267.83 & 2936.82 &                     	4114.66 & 1072.92 &	2205.82 & 3260.33 \\\cline{2-8}
& mixed & 3822.99 & 3427.06 & 4114.80 & 2145.75 &	2205.83 & 3260.31 \\\hline
\multirow{3}{*}{FSMDriver3}
& road & \textbf{7925.60} & 13196.50 & 8745.49 & 3978.01 & 3451.26 & 6757.83 \\\cline{2-8}
& dirt & 2149.77	& 2450.84 & 1951.93	& 3525.84 & 4905.58 & 5590.78 \\\cline{2-8}
& mixed & 7219.28 & 12772.40 & 8126.12 & \textbf{4386.64} & \textbf{5481.15} & \textbf{6939.83} \\\hline
\multicolumn{2}{| c |}{AUTOPIA~\cite{AUTOPIA2009}} & 7091.8 & \textbf{15612.3} & \textbf{8970.4} & * & * & * \\\hline
\end{tabular}
\end{table*}

Table~\ref{tbl:ticks} presents the results for these tests for each FSMDriver model and profile. The fitness function in this case is the sum of distances covered in all tracks, considering 10.000 game ticks per track, as per the SCR rules. The table also includes, for direct comparison with the current state of the art, results for AUTOPIA in three of the same tracks for the same 10.000 game ticks metric, as presented in~\cite{AUTOPIA2009}. The `*' symbol indicates that the driver does not have data available for the track.

Considering the 5-states FSMDriver in road tracks, it is clear that the road and mixed profiles have better results than dirt, as expected. However, in \emph{CG Speedway 1} they are practically the same. \gnramos{por que?} Looking at the results for dirt tracks, all profiles have similar performances except for the dirt profile in Dirt 1 \gnramos{por que?}

Considering the 3-states FSMDriver in road tracks, it is also clear that the road and mixed profiles have better results than dirt, as was expected.  Looking at the results for dirt tracks, the mixed profile's performance was significantly better than the others. FSMDrivers evolved in dirt tracks tend to drive slower to avoid leaving the track limits while evolution in road tracks leads to faster racers which may lose time off track. The combination of both tracks in the metric leads to drivers that compromise in both approaches, being about 5-10\% slower than the road profile in road tracks, and about 5-15\% faster in dirt tracks,a more general driver as hypothesized. It is also better than the dirt profile on all accounts.

Comparing the results with AUTOPIA, the road FSMDriver3 was about 11.7\% better in \emph{Street 1}. By itself this does not imply that FSMDriver3 is a better approach, specially since AUTOPIA had better results in two of the three tracks compared, it still is an impressive result for this simpler approach to driving.

In order to learn more about FSMDriver's potential, a new set of tests using the distance raced in 10 laps as metric was done, and the results are shown in Table~\ref{tbl:time}, where the `\textdagger'~symbol int Table~\ref{tbl:time} indicates that the driver does not complete the 10 laps required, and the `*' symbol indicates that the driver does not have data available for the track. Results for AUTOPIA in three of the same tracks for the same 10 laps metric, as presented in~\cite{AUTOPIA}.

Table~\ref{tbl:time} also includes results for the robot \emph{Berniw Hist4}, provided in with the TORCS distribution. This was not included in the previous tests because you can only configure a race with the available robots to run for a given amount of laps (not game ticks). This robot was chosen to provide insights on the controllers performance compared to a robot's, having already been compared to AUTOPIA~\cite{AUTOPIA}. \emph{Berniw Hist4} does not, however, use the standard car \emph{car1-trb1}, it uses the \emph{TZ2}.

\begin{table*}[t]
	% \renewcommand{\arraystretch}{1.3}
\caption{Time in seconds elapsed racing alone for 10 laps}\label{tbl:time}
\centering
\begin{tabular}{| c | c | c | c | c | c | c | c |}
	\hline
\multicolumn{2}{| c |}{\bfseries Driver} & \bfseries Street 1 & \bfseries D-speedway & \bfseries CG-speedway & \bfseries Dirt 1 & \bfseries Dirt 3 & \bfseries Dirt 4 \\\hline
\multirow{3}{*}{FSMDriver5}
& road & \textdagger & \textdagger & 816.6 & \textdagger & \textdagger & \textdagger \\\cline{2-8}
& dirt & \textdagger & \textdagger & \textdagger & \textdagger &	\textdagger & \textdagger \\\cline{2-8}
& mixed & \textdagger & \textdagger & \textdagger & \textdagger & \textdagger & \textdagger \\\hline
\multirow{3}{*}{FSMDriver3}
& road & 1086.30 & 607.40 & 495.00 & \textdagger & 1150.10 & 1756.40 \\\cline{2-8}
& dirt & \textdagger & 840.00 & 1274.80 & 597.30 & 1089.40 & 1307.50 \\\cline{2-8}
& mixed & 1216.50 & 572.60 & 613.77& 530.00 & 842.90 & 1005.90 \\\hline
\multicolumn{2}{| c |}{Berniw Hist4} & 1143.77 & 656.24 & 605.76 & 460.95 & 872.97 & 1127.45 \\\hline
\multicolumn{2}{| c |}{AUTOPIA~\cite{AUTOPIA}}  & * & * & * & \textbf{339.3} & \textbf{742.4} & \textbf{796.5} \\\hline
\end{tabular}
\end{table*}

Looking at results for FSMDriver5, only the road profile completed 10 laps in \emph{CG Speedway 1}, all other instances had excessive damage. FSMDriver3, on the other hand, completed all track in at least 2 configurations. Considering road tracks, the road profile had better performance in two of the three tracks, and the dirt profile did not complete one but was significantly slower in two of the tracks. Considering the Dirt tracks, again the mixed profile's performance was significantly better than the others.

Considering the robot \emph{Berniw Hist4}, it can be seen that FSMDriver3 is a slightly better driver, despite the ``handicap'' of using SCR's interface for driving and not having access to additional information from the simulator (such as track layout). This means that the proposed model has quickly achieved an intermediate level performance in TORCS. AUTOPIA's results are around 12\%-35\% better than FSMDriver's best results, implying that its configuration is better suited for longer races.

This is likely due to lack of endurance, resulting from careless driving. Since dirt profile tends to drive slower, it drives farther on this longer test.





% \subsection{Analysis} \label{subsec:Analysis}

%	The overall comparison between the two approaches presented favored the Three-State FSM, on account of the considerably superior results it produced in all the tracks tested. The Five-State FSM presents a very complex transition function that takes into account the variance of the sensorial input to decide if the car is in a straight line, approaching a turn or in a turn. On the other hand, the Three-State FSM has only a single state for handling normal driving situations and it is very easy to say if the controller is inside or outside the track only by checking the track sensor. By having multiple states acting inside the track the Five-State FSM leads to a struggle in defining the boundaries of a curve, the characteristics of an approaching curve and how those situations are different from a straight line. Mismatching emergent from the transiction function would often cause the car to leave the track. It seemed that a single state for handling those situations ended in more accurate behavior on account of having no dependence from an external function.

%	Besides the difficulty in deciding on which state should be triggred. There is the complexity related to defining the behavior for each state. The steering control in the straight line was too smooth and sometimes depending on how the car left a curve it was not capable of correcting the trajectory and could led to those exception situations. However if the same control was too abrupt like the curve's one it would also led to exception situations due to excessive steer. These and other problems emphasized the need of a unified state for dealing with those situations.

%	This contrast in behavior is then interpreted to be the reason of the overwhelming difference in performance, the complexity of the transition function, which supports the initial hypothesis of the evaluation. Due to this attribute, the Five-State FSM undergoes a lot of damage in its car, which can be noted in Table~\ref{tbl:time raced} where all the ``\textdagger'' symbols represent individuals that did not finish the race for the reason of reaching the maximum damage permitted.

%	Because the Three-State FSM demonstrated better results than the other approach proposed, it was elected to be subject of analysis on the evaluation process. As expected, the controller evolved only on Road Tracks was the fastest one. These tracks provide an environment susceptible to high speeds, since its curves are smoother and the friction experienced by the car is higher than the ones from Dirt Tracks. These factors, when combined, allow the controller to race without having to steer too abruptly and to brake without losing control while racing in Road Tracks. Consequently, as the friction increases, steering becomes more accurate in road tracks, practically eliminating critical skidding. Therefore, the result from this end of the evolution process was an aggressive driver with high base-speed.

%	Dirt Tracks provide a more difficult environment for the pilot to fit in. Sudden braking in tracks of this type often results in unwanted behavior, skidding is noticeably more common then. The driver evolved in this end of the evolution process tends to drive in a low speed so it can keep itself inside the boundaries of the track. Speed driving results in higher damage outcomes and even in the total loss of the car in critical situations. The result obtained was a very careful driver with a low base-speed, and an early brake policy - the car starting to brake far before the turn. This passive driving pattern obtained the smallest distance covered both for the Three-State and the Five-State FSMs.

%	The driver evolved in a mixed set of tracks combines characteristics from both of them. It drives in a reasonable speed comparing to the first one, but also has the preventive brake policy from the second one. This last end of the evolution process achieved better results than the Dirt evolved behavior in all the tracks tested and outperformed the road-evolved one in every single dirt track. From this information gathered, it was inferred that the controller evolved in mixed tracks tries to reach higher speeds even though this means leaving the track in some turns, mostly because the time spent trying to get back to the racing lane is compensated by the speed of the car. The aggressive behavior inherited from the road-evolved end of the evolution makes this latest controller receive ample damage when leaving the track, and also causes it to hit walls, which resulted in the premature ending of some of the tested races, due to reaching the maximum acceptable damage.

% \subsubsection{Comparing the Three-State FSM to AUTOPIA} \label{subsubsec:CompAUTOPIA}

%	Once the Three-State FSM was demonstrated to be more suitable to competitive environments due to its superior performance regarding the Five-State FSM, it was compared to the renowned controller AUTOPIA. Using the distance covered after racing alone in Road Tracks for 10 000 game tics as metric, the Three-State FSMs evolved in Road Tracks and in mixed tracks were able to overcome AUTOPIA in 1 of the 3 tracks tested, as displayed in the bold values in Table~\ref{tbl:dist covered}. The road-evolved Three-State FSM was the controller that got closer to this State of the Art approach using the ``distance raced'', which comes to endorse the assumption of it being a competitive proposal.

%	However, while racing alone for 10 laps and computing the time elapsed as metric, AUTOPIA outperformed every controller proposed, just as can be seen in Table~\ref{tbl:time raced}. Even though the Three-State controller with Road evolved parameters outperformed AUTOPIA in the first 10 000 tics it was not capable of maintaining the advantage in longer races. The road evolved pilot presents a more aggressive behavior even though it means taking more damage it has a gain in performance for the early stages of the race. Although when racing for more than a couple laps the controller becomes more careful after each lap reducing it speed to maintain itself inside the track.

%	The graphical analysis was quite useful at this point. It revealed that AUTOPIA's brake, acceleration and recovery policies are robuster than the ones presented in this papper. AUTOPIA barely leaves and track and when it does a fast recovering behavior is performed  resulting in small losses in performance. It also has a better stability control which can be observerd moslty in Dirt tracks, where the car is more susceptible to splip and skidding.

%	The online learning module plays a crucial role in the overall controller's performance as it prevents unwanted situations to repeat, for example leaving the track. Although this strong dependence might result in performance loss as the controller will gradually reduces it speed after each lap in those points where it leaves the track. More accurate actuators control may reduce the dependence of this module and therefore improves performance.

%	These results can be used to infer that the Five-State and the Three-State FSMs have a great deal of improvement to achieve when it comes to endurance. The Five-State FSM received total loss and did not complete almost every test performed, ending only one race using this metric. The Three-State FSM, on the other hand, completed practically all the tracks, but did not surpass AUTOPIA in either of them. In order to enhance the endurance feature in the controllers proposed, more robust behavior concerning situations in which the car might crash must be taken into account.


\subsection{Analysis}
Further comparing both FSMDrivers, it is clear that different profiles lead to different performances on the tracks for FSMDriver3, but such effect is not as pronounced in FSMDriver5. A more detailed analysis shows that FSMDriver5 has issues handling transitions in \state{Racing} behavior. The issue seems to be that the parameters \gnramos{???} in the transition function that define the current stated are very susceptible to the inputs, resulting in several changes in the states which greatly impacts the evolution of the states' parameters and, consequently, of those in the function.

This leads to two possibilities, the first being that the specified settings for evolving FSMDriver5 were not adequate, perhaps allowing more generations or individuals would lead to better results. More likely, the 5-state model's implementation is overly complex, specially considering the influence of the transition function's parameter \gnramos{parametros??} on how the states evolve, which could imply that a different strategy should be pursued.%
	\section{Conclusions}\label{sec:5}
Self-driving cars are one of the most interesting promises of technology, but the process of developing a software controller  is complex and expensive. Advanced car racing simulators present a well defined environment for developing such drivers and comparing Artificial intelligence solutions for specific problems such as path planning~\cite{deFreitas:2012}, controlling human non-playable character~\cite{simon2008}, car racing~\cite{2009}, among others.

The Open Racing Car Simulator (TORCS) is a modern, modular, highly-por

also complex since testing is a frequent task and real driving systems are expensive. Creating and testing solutions can be aided by realistic car racing games which can closely simulate the roads, other vehicles, and their complex interactions~\cite{caldeira2013torcs}. Games also present a well defined environment which may be used not only for applications of machine learning results, such as neuroevolution~\cite{stanley_real-time_2005,5482132} or human pose recognition~\cite{Shotton:2011}, but also for comparing AI solutions for specific problems such as path planning~\cite{deFreitas:2012}, controlling human non-playable character~\cite{simon2008}, car racing~\cite{2009}, among others.

The Open Racing Car Simulator (TORCS) is a modern, modular, highly-portable multi-player, multi-agent car simulator~\cite{SIMUTORCS}, one of the most advanced racing games available, and frequently used as a platform for comparing driver solutions in the Simulated Car Racing (SCR) Championship~\cite{2009,Loiacono:2012:LEA:2212908.2212953}. This paper uses a finite-state machine (FSM) controller to exploit the advantages of a divide-and-conquer approach by considering the task's various situations as distinct states (racing, getting back on track, etc.).

FSM is a well-known mathematical model with widespread applications such as controlling air conditioning systems~\cite{BERNARD}, highway surveillance systems~\cite{DOHYUN}, and even simulated car racing~\cite{DIEGO}. Such solutions are implemented as software and configured according to a specified set of parameters, and defining the best possible configuration is usually a huge combinatorial problem for which exhaustive systematic searches become unfeasible.

Autonomous vehicles have many practical applications, however the development of software controllers for such task is difficult. This work presents a finite state-machine model with evolved parameters as a suitable solution for a self-driving car, and the comparison of two different configurations in The Open Racing Car Simulator. This approach enables a clear division of behaviors in states, providing an easy way to test different configurations and simplifying the search for better controllers by allowing changes in selected states. A 5-state and a 3-state drivers were evolved through genetic algorithm and compared to each other and to AUTOPIA, the current state of the art controller for the Simulated Car Racing Championship. Results showed that the proposed model has potential for racing, even surpassing one of AUTOPIA's marks, and provide insights on developing and configuring.

	This paper proposed two approaches developed to control a car during a race in a simulated computational environment, the game platform TORCS. The models of both of these controllers were described, explained, enhanced by means of a genetic algorithm, compared and then tested together with a State of the Art controller - AUTOPIA. It was implied before the testing phase that a finite state machine too burdened in the process of transition between states might lose performance, which was corroborated by the experimental results of the comparison of the two models detailed.

\subsection{Conclusions} \label{subsec:Conclusions}

	A veil was put over the Five-State FSM from the very beginning of the experimentation phase, which was its great dependency towards the transition function. Early superficial evaluations of the performance of this first model indicated an overcharge concerning this function, which was verified by considerable changes in behaviour derived from adjustments in its parameters. For that reason, a second model with less states - the Three-State FSM - was designed regarding this characteristic and releasing part of the performance burden from the transition function.

	The finite state machine with less states achieved a superior overall performance in the tests carried out, in relation to the one with more states. The simplifications fashioned in the transition function of the former were inferred to be the reason for this improvement, along with its intricate relation with the number of parameters that were target of fine tuning in the evaluation and validation process.

	The evolution procedure adopted concerning the controllers culminated in three characteristic behaviors. The controller evolved on road tracks became a fast driver, whose hastiness resulted in a careless attitude in general; in other words, it was only good for races in road tracks. The one evolved on dirt tracks turned out to be too careful in contrast, limitedly determining its speeds and, in efficiency terms, inferior. The controller evolved on a mixed set of tracks inherited characteristics from both the previous ones, becoming swift but not too hasty, prudent but not too slow. The latter surpassed the performance of the dirt-evolved drivers even on dirt tracks, and did not lose by far on road tracks in comparison to drivers evolved solely in them.

	In terms of speed, the Three-State FSM was able to overcome AUTOPIA in one of the three tracks used to validate the drivers using the distance covered in 10 000 game tics as metric. However, in comparison to the same controller, using the time elapsed to race 10 laps as metric instead, the experimental results provided an insight on the lack of endurance that the finite state machine drivers proposed possess.

	To sum up, the interpretation of the global results from the experiments performed gives margin to declare that finite state machines are a reliable technique to implement artificial intelligences, at least for computer games such as simulated races. They provide the possibility of parallel development and also enable parameter tuning in separate fronts, due to the independence and abstraction between the behaviors from each state. Finite state machines also represent a valuable tool for describing an operation model for a process, simplifying and gathering possible situations it might present into straightforward categories of accessible understanding.

\subsection{Future Works} \label{subsec:Future}

	One characteristic that has been marked as a deficiency in the controllers presented in this paper is the lack of endurance. In order to prevent this from affecting the general performance of the controllers developed, more robust techniques must be integrated into their model. Ways of treating this matter range from harsher brake policies to drive planning intensification, which are already being taken into account for future proposals.

	Anti-lock Breaking System (ABS) filter is commonly used in the racing environment~\cite{5593318} and was not implemented in the proposed controller. The Traction Control System (TCS)~\cite{5593318} could also be implemented by applying filters to the acceleration actuator This policies minimize skidding and increase the stability when speed needs to be adjusted in a turn.

	The driver uses some parameters that are necessarily greater than others. A model based on fuzzy logic \cite{DIEGO} could simply solve this ordering issue.

	Another important task to be accomplished is the opponent treatment in real-time races. Routines to reduce collisions, i.e., avoiding being overtaken and also being concerned about overtaking the opponents is a fundamental issue. Ignoring adjacent cars usually causes the driver to face unexpected collisions, ending up stuck, considering a worst case scenario. Many of the renowned developers for TORCS already incorporate such treatment in their controllers, and neglecting this necessity renders any driver less robust to unexpected race events, and also reduces its performance.

%
	\FloatBarrier%

	%\section*{Acknowledgements}

	\bibliographystyle{sbgames}
	\bibliography{Bibliography}

\end{document}
