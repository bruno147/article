\section{\textbf{The FSMDriver}} \label{sec:FSM}
	
	Some of the main aspects discussed to outline a controller for TORCS were sufficed by the concept of finite state
	machines (FSM)~\cite{Millington:2006:FSM}, the most important one being the goal of reaching an autonomous driving
	behaviour in a car race.

	\toDo{o quotation, definicao de fsm, deve estar aqui, fora do five-state FSM}
	
\subsection{Five-state FSM} \label{subsec:FSM5}
	
	According to Mat Buckland in \emph{Programming Game AI By Example}~\cite{Buckland:2005:AI}:
	
	\begin{quotation}
		
		\emph{
		``A finite state machine is a device, or a model of a device, which has a finite number of states it can be in
		at any given time and can operate on input to either make transitions from one state to another or to cause
		an output or action to take place. A finite state machine can only be in one state at any moment in time.''}
		
	\end{quotation}
	
	This architecture was chosen in order to transform the problem of complex driving into smaller problems of
	situations found within the racing environment. Initially, the design of the finite-state machine proposed
	comprised the following states:
	
	\begin{itemize}

	\item \emph{Straight Line};
	
	\item \emph{Approaching Curve};
	
	\item \emph{Curve};
	
	\item \emph{Out of Track};
	
	\item \emph{Stuck}.

	\end{itemize}
	
	Essentially, for this method, normal behaviour covered \emph{Straight Line}, \emph{Approaching Curve} and
	\emph{Curve}, as the controller was located inside the track boundaries and no recovery actions needed to be
	considered, whereas exception behaviour consisted of \emph{Out of Track} and \emph{Stuck}, situations in which
	such conduct was expected. Here, a consideration needs to be taken into account: the real first model of the
	finite-state machine did not have an \emph{Approaching Curve} state, but, as the interpretations of the demeanour
	concerning the \emph{Straight Line} and the \emph{Curve} were so different from one another, a preparation had to
	be established so as to smoothen the transitions between them.
	
	If the controller was currently in \emph{Straight Line}, it would be expected of him to simply go as fast as he
	could, with no steering changes whatsoever; when in \emph{Approaching Curve} state, he would reposition himself
	in relation to the track and recalibrate his speed in order to make better curves, which is done by steering the
	car towards the direction of the sensor with the biggest read value and braking until a proportionally calculated
	target speed is achieved; or, when in situations of \emph{Curve}, the pilot would stop braking while maintaining
	the steering direction towards the sensor pointing the biggest distance value read - which represents the
	direction of the curve, prepared by the approaching curve state. For the exception states, even though the
	expected deportment is well known, e.g. a stuck controller should maneuver the car out of the current situation
	and proceed to normal race conduct, the implementations vary among developers. The strategy chosen for the
	exception states was used on both proposals as a matter of regularity of comparison, and is explained on the next
	subsection.
	
	One big problem about this way of treating the matter is that the function responsible for choosing which state
	is more appropriate for each situation would more than often be overcharged, and, in some cases, rather different
	sets of parameters received by it would result in the same classification among the states. Thus, in order to
	minimize the dependency of the driving performance in relation to the function in charge of the transition
	between states, a project decision was made to reduce the number of states.
		
\subsection{Three-State FSM} \label{subsec:FSM3}
	
	The very nature of the initial architecture gave rise to the new approach. As mentioned within this very section,
	three of the states were innately part of a common bigger state, thereby \emph{Straight Line},
	\emph{Approaching Curve} and \emph{Curve} summed into \emph{Inside Track}, resulting in the new controller with
	only three states, which were:
	
	\begin{itemize}
		
		\item \emph{Inside Track};
		
		\item \emph{Out of Track};
		
		\item \emph{Stuck}.
		
	\end{itemize}
	
	\emph{Inside Track}, therefore, is how the car, desirably, will spend most part of the time. The controller
	calculates a \emph{target speed} based on how far the car is from the farthest edge of the track, then, it assumes
	a position of increasing the speed until it reaches this velocity while driving towards the sensor with the
	biggest read value. The distance of this method conveys the greater length the car may advance with little or
	sometimes without significant steer changes This state also brakes, if necessary, when approaching turns.
		
	\emph{Out of Track} is when, for any unknown reason, the car is found outside of the track limits. In this case,
	the proper behaviour is to try to return to the lane. In road tracks, the outside track normally has a different
	terrain, sometimes dirt-based, meaning that skidding frequently occurs, and in an effort to avoid this, a control
	system to brake when the car begins skidding above a threshold was implemented.
	
	\emph{Stuck} represents any given situation that the car is unable to progress in the race. This is a delicate
	state, because it presents itself as difficult to identify and also due to its impact to the performance of the
	controller. In order to detect \emph{Stuck} circumstances, the speed of the car is monitored throughout the race,
	during every game tick, if it lingers with a low speed for a determined period, then it is considered stranded,
	or stuck. When detected, this state activates the reverse gear of the car and turns it until its front is
	directed towards the correct axis of the track. The reason why \emph{Stuck} is a sensitive state is because, when
	detected early, might indicate false positive, and when detected late, could lessen the efficiency of the
	controller. Thusly, detecting \textit{Stuck} situations is crucial, and so is handling the car out of them.
	
	Besides the states, there is also a learning module that is called whenever the \emph{Out of Track} state is
	requested. This module records both speed and position from the state of the car in the vicinity of the departure
	from the track, and retaining these information allows the controller to slow down in subsequent laps when
	approaching the critical points highlighted by the learning module. The implementation of this procedure consists
	on replacing the speed recorded for the landmarked position by a slower one on future occurrences, which is done
	by starting to break in these next occasions. For this work, the module described in this paragraph constitutes
	the Online Learning method, concept introduced in Subsection~\ref{subsec:Related}.
	
	In conclusion, each state in this manner of dealing with the process becomes, ideally, an independent problem,
	whose solution can be attacked separately. This way, they can all have individual sets of parameters susceptible
	to improvement, which will be discussed in the next section.
	
\subsection{Search for Parameter Values - Genetic Algorithm} \label{subsec:GA}
	
	Due to the quantity of parameters to be tuned and the defined granularity, the search space becomes enormous and
	renders fundamental the use of a search algorithm, which optimizes the process of finding better configurations
	by being more incisive and saving resources such as computational time and space. In the present study, an
	evolutionary algorithm was chosen for this task.
	
	A genetic algorithm~\cite{GA} is an evolutionary algorithm inspired by nature, in special by the concept of
	evolution through natural selection~\cite{Darwin}, whose main idea is that a set of solutions for a problem can
	be evolved like the population of a generic species in nature. The applications of genetics algorithms are
	present at many areas, extending from control engineering at non-linear system identification to biomedicine
	prosthesis development and even economy to forecast the behavior of agents.\toDo{referencias sobre aplicacoes}
	
	\toDo{trecho}In this context, a viable solution is called an individual and can be represented by a string of parameters; each
	individual then becomes an array that contains the information necessary to evaluation.\toDo{ate aqui, tem duas frases iguais} The first population is
	instantiated randomly to its full extent, because the search space is too big for an initial attempt at a good
	possibility\toDo{, nao eh porque eh grande, mas simplesmente porque nao temos nenhuma dica inicial}. This population passes through a fitness function that indexes a score to each individual. This
	function is responsible for assessing how good - or how adapted - the solution that is being evaluate is. After
	evaluating the population separately, a group of individuals is chosen as the parents of the next set of
	solutions, which will compose a new generation; there are countless ways of performing the selection of the
	parents to the new generation of offspring, and this work gave preference to picking the higher individuals on the
	scoring system, what is called Elitism~\cite{ELITISM}. Each pair of parents is submitted to
	crossover in order to generate two offspring solutions, and in the end of the process each offspring may
	present mutation - everything according to predefined rates. For this work, the crossover takes place in 95\% of
	the reproduction, while the mutation rate assumes the rate of 1\%~\cite{RATES}.
	
	For the Five-state FSM, 22 parameters required adjustment, which originated, in different quantities, from each
	state separately and also the \emph{transition} function, as follows:
	
	\begin{itemize}
		
		\item \emph{Approach Curve} has 4 parameters;
		
		\item \emph{Transition Function} has 3 parameters;
		
		\item \emph{Straight Line} has 4 parameters;
		
		\item \emph{Out of Track} has 7 parameters;
		
		\item \emph{Stuck} has 4 parameters.
		
	\end{itemize}
	
	However, for the Three-state FSM, only 19 parameters demanded adjustment, which are divided as follows:
	
	\begin{itemize}
		
		\item \emph{Inside Track} has 8 parameters;
		
		\item \emph{Out of Track} has 7 parameters;
		
		\item \emph{Stuck} has 4 parameters.
		
	\end{itemize}
	
	Theoretically, as the search space for the latter case is smaller, finding a better controller was expected to
	happen first for it, but only by effectively testing both architectures could such a result be achieved. The
	source codes for both models presented in this section are available at the \emph{GitHub} repository provided in
	the references~\cite{GitHub}.
	