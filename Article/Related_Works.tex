\section{\textbf{Related Works}} \label{sec:relworks}
	
		The reason why TORCS presents itself as a satisfactory AI benchmark is because there is an infinity of
		possibilities on how the sensorial input received from the server will be translated into the behaviour of the
		actuators, and they can all be compared in a race, which has a robust and steady scoring system. In other
		words, there are many different approaches concerning how to teach the racer encoded by the developers to
		drive in a racing competition only with the information given by the sensors, and the metric to that issue
		is the performance on the race itself.
		
		By controller, let it be understood that the subject is the programming code that in fact controls the
		car/driver/racer within racing environment. Some examples of awarded controllers and their driving methods
		will be presented in this section. They are what can be called the State of the Art among TORCS, and it is
		very common among them the incorporation of machine learning methods, along with other evolving techniques
		using artificial intelligence. Instinctively, as the nature of the problem comprises evolution by experience,
		learning procedures tend to enhance performance and competitiveness. Essentially, there are two ways of
		evolving controllers: Online Learning and Offline Learning, the first meaning that improvements are achieved
		during the actual race execution time and the latter that it is done before the competition, on the account of
		the developers themselves and with their own resources.
		
\subsection{State of the Art}
		
		The current champion of the SCR Championship is the controller \emph{Mr. Racer}~\cite{MrRacer}, and it has
		proven to be the State of the Art by winning at least the last three competitions that happened. The authors
		of this implementation learn parameters offline through Covariance Matrix Adaptation Evolution Strategy
		(CMA-ES), use regression and low-pass filtering to reduce noise impact, distinguish normal asphalted roads
		from dirt-based ones for behavioral separation and implement an authentic opponent-handling method. Their
		Online Learning consists on the track model selection to categorize into dirt or asphalt, choice of databased
		sets of parameters that best fit the track and the tuning of a target speed for all its corners.
		
		Another renowned controller is \emph{AUTOPIA}~\cite{AUTOPIA}. According to the founders of the
		competition~\cite{SoA} and the authors of \emph{Mr. Racer} themselves, it is a competitive match, with the
		potential to even be the best one available, but since no entries were received from them in a while, their
		means of winning a competition were somewhat restrained. Nevertheless, assessing its performance is
		worthwhile, and its description is the implementation of a modular Fuzzy Architecture, whose division contains
		gear, steering and speed control. Their controller is optimized by means of a genetic algorithm for Online
		Learning, and by means of landmarking the lane exit points for further speed reduction for Offline Learning.
		
		\toDo{Nós vamos	comparar o nosso piloto com o AUTOPIA? Se sim, devemos dizer isso aqui.}
		
		These and other controller exemplifications served as parameters for the analysis and development of the
		approach presented in this paper. Aspects incorporated and adapted feature modularity, offline learning
		through genetic algorithms, online learning through landmarking and choosing sets of parameters for different
		categories of tracks, etc. \toDo{Adicionar o quê mais fizemos no projeto como pincelada inicial para fazer o
		gancho com a seção Controller Structure.}