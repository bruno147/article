\section{\textbf{Introduction}}
	
	Automation of day to day tasks is an endeavour that has moved a large amount of scientific resources in the
	recent history. One specific example is target of studies around the globe by a lot of universities, companies
	and industries, which is the automation of vehicles, more specifically, automobiles. The objective of such
	attempts is the development of artificial intelligences capable of driving a car safely, with traffic law
	enforcement, real-time decision making, efficiency and, in addition, resource economy - as with gas or even
	time. The practical applications of such controllers in autonomous vehicles are enormous, for example, the 
	DARPA\footnote{http://www.darpa.mill} challenge. \toDo{Deixar do jeito que está ou colocar na Bibliografia?}

	The Simulated Car Racing Championship (SCRC), using the platform TORCS (The Open Racing Car Simulator), has
	brought an excellent environment for benchmarking AI approaches for the problem of autonomous car controllers.
	Notwithstanding, the optimum behaviour of a controller is a complex matter in its full extent; for this purpose,
	the strategy adopted to deal with it was to divide the problem into smaller portions, i.e., less complicated
	subproblems, in order to implement a finite-state machine that admittedly covers all necessary behaviours. Later
	on this paper, each of those subproblems are treated as the states of the referenced finite-state machine, which
	individually incorporate different but complementary parts of the integral behaviour.
	
	Computers are more suited for applications that require testing different sets of parameters or configurations for
	a given problem, and although hand-coded methods might present satisfactory outcomes in questions of such
	intricacy such as the one discussed in this paper, they will hardly ever outperform the ones that are
	computer-aided. Considering this, after an initial structure of the controller was designed, a method of fine
	tuning was assimilated to it, which was a genetic algorithm.
	
	Apart from this Introduction, this paper is structured as follows: Section II introduces TORCS, the working
	environment used in the context of this task, along with the competition, SCR Championship, that currently
	represents the utmost metric to evaluate the performance of the controller proposed; Section III presents what is
	already being done at this context in related works, highlighting the strategies that are standing out and
	analyzing the characteristics responsible for it; Section IV then explains the proposal of the developed
	controllers, clarifying their behaviour and structure; Section V defines how these proposals were improved by the
	computer-aided method of a genetic algorithm; Section VI describes how the validation process occurred, through
	the methodology, the experiments and the results achieved; Section VII analyzes the results originated from
	Section VI and provides conclusions about them, which establish the comparison between the finite state machines
	with few and with moderate number of states; and Section VIII points out prospects about what is yet to be done
	in future works.
	
	A genetic algorithm is inspired on nature, in special with the concept of natural selection introduced by Charles
	Darwin. The main idea is that a individual can be represent in a form of strings of parameters, and a randomly
	first population is created, pass through a fitness function that index a score to each individual of the
	generation, then that parents to the next generation is chosen in function of their scores, meaning that the
	better ones have more chance to spread their chromosome. \toDo{Colocar na seção de Algoritmo Genético.}