\section{\textbf{TORCS Environment and SCRC}} \label{sec:torcs}

\subsection{TORCS as platform, interface and environment}

	The Open Racing Car Simulator (TORCS)~\cite{TORCS} is a widely used platform for benchmarking AI, renowned for
	its	highly credible physics modeling engine and yet user-friendly interface for car racing simulation. One of
	the	many other qualities of this open-source simulator is its portability, concerning multi-platform environments
	- such as different operational systems and architectures - and multi-programming language ones alike.
	
	TORCS deals with a robotic simulation in the format of car racing. It serves both as an ordinary AI game and a
	research platform, for it provides a complete sensor-based interaction system in which the developer is able to
	interpret received parameters of the car - such as speed in X and Y axes - and construe them in order through
	coding to control the car through actuators, some of which are acceleration and steering. This behaviour is
	widely embraced when it comes to artificial intelligence methods - a set of inputs to result in a set of outputs.
	
	Another credibility factor for this platform is its non-punctual cars, each of which possess a body with sensors
	and actuators, as mentioned earlier, and interact with other cars in the environment by a life-like collision
	system. Nevertheless, TORCS is still a simulator, and its limitations, along with the defined racing environment
	and the modeled robotic car, are more than likely to affect any results obtained. This is an inherited
	characteristic of any real-life problem simulation, what in academia is denominated \emph{reality gap}, and it
	stems from the simplifications attained concerning the car models, the technical features of the tracks, and so
	forth.

\subsection{The SCRC}

	The Simulated Car Racing Championship (SCRC)~\cite{SCRC} is an example of a well-known competition that utilizes
	TORCS as interface. Being an event joining three competitions held at major scientific conferences, such as
	\emph{IEEE Congress on Evolutionary Computation}~\cite{CEC}, \emph{Genetic and Evolutionary Computation
	Conference}~\cite{GECCO} and \emph{IEEE Conference on Computational Intelligence and Games}~\cite{CIG}, it is
	an accepted	metric of evaluation in the fields of Evolutionary Computation and Computational Intelligence
	regarding Games.
	
	The SCRC features encapsulation with its interface towards TORCS, meaning that there is an abstraction layer
	between the two and, in addition, some of the information about the racing execution remains hidden from the
	controllers, such as the geometrical format of the track and its category. SCRC and TORCS communicate through
	UDP packages, and each player receives information regarding the sensors of the car, some of which were already
	presented earlier on this section.
	
	A brief description of the sensors is as follows: the driver can use a set of sensors that inform relative
	position, speed, and car condition such as rpm, damage and gas. Useful information about the car during a
	race can be mined from the data provided by them; for example, there are two sets of sensors labeled
	\emph{track} and \emph{trackpos} that inform the position of the racer according to a desired direction and
	according to the track track axis, respectively, and these two sets may be combined to inform the absolute
	location of the car.
	
	The complete sensorial input information can be found at the Simulated Car Racing Championship Competition
	Software Manual~\cite{SCRC}, within section 7.7. Noise can be incorporated to the aforementioned sensors,
	option that is inexorably present during the actual competition and will be dealt with in the section Future
	Works.
	
	The races from the SCRC adopt the following structure: Warm-up, when each driver is able to explore the track
	and deduce information from it at will, for a limited time; Qualifiers, a stage that places each driver in a
	race alone against the clock; and the Actual Race, when finally the eight best pilots from the Qualifiers race
	competitively.